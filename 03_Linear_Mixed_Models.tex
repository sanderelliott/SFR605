% Options for packages loaded elsewhere
\PassOptionsToPackage{unicode}{hyperref}
\PassOptionsToPackage{hyphens}{url}
\documentclass[
]{article}
\usepackage{xcolor}
\usepackage[margin=1in]{geometry}
\usepackage{amsmath,amssymb}
\setcounter{secnumdepth}{-\maxdimen} % remove section numbering
\usepackage{iftex}
\ifPDFTeX
  \usepackage[T1]{fontenc}
  \usepackage[utf8]{inputenc}
  \usepackage{textcomp} % provide euro and other symbols
\else % if luatex or xetex
  \usepackage{unicode-math} % this also loads fontspec
  \defaultfontfeatures{Scale=MatchLowercase}
  \defaultfontfeatures[\rmfamily]{Ligatures=TeX,Scale=1}
\fi
\usepackage{lmodern}
\ifPDFTeX\else
  % xetex/luatex font selection
\fi
% Use upquote if available, for straight quotes in verbatim environments
\IfFileExists{upquote.sty}{\usepackage{upquote}}{}
\IfFileExists{microtype.sty}{% use microtype if available
  \usepackage[]{microtype}
  \UseMicrotypeSet[protrusion]{basicmath} % disable protrusion for tt fonts
}{}
\makeatletter
\@ifundefined{KOMAClassName}{% if non-KOMA class
  \IfFileExists{parskip.sty}{%
    \usepackage{parskip}
  }{% else
    \setlength{\parindent}{0pt}
    \setlength{\parskip}{6pt plus 2pt minus 1pt}}
}{% if KOMA class
  \KOMAoptions{parskip=half}}
\makeatother
\usepackage{color}
\usepackage{fancyvrb}
\newcommand{\VerbBar}{|}
\newcommand{\VERB}{\Verb[commandchars=\\\{\}]}
\DefineVerbatimEnvironment{Highlighting}{Verbatim}{commandchars=\\\{\}}
% Add ',fontsize=\small' for more characters per line
\usepackage{framed}
\definecolor{shadecolor}{RGB}{248,248,248}
\newenvironment{Shaded}{\begin{snugshade}}{\end{snugshade}}
\newcommand{\AlertTok}[1]{\textcolor[rgb]{0.94,0.16,0.16}{#1}}
\newcommand{\AnnotationTok}[1]{\textcolor[rgb]{0.56,0.35,0.01}{\textbf{\textit{#1}}}}
\newcommand{\AttributeTok}[1]{\textcolor[rgb]{0.13,0.29,0.53}{#1}}
\newcommand{\BaseNTok}[1]{\textcolor[rgb]{0.00,0.00,0.81}{#1}}
\newcommand{\BuiltInTok}[1]{#1}
\newcommand{\CharTok}[1]{\textcolor[rgb]{0.31,0.60,0.02}{#1}}
\newcommand{\CommentTok}[1]{\textcolor[rgb]{0.56,0.35,0.01}{\textit{#1}}}
\newcommand{\CommentVarTok}[1]{\textcolor[rgb]{0.56,0.35,0.01}{\textbf{\textit{#1}}}}
\newcommand{\ConstantTok}[1]{\textcolor[rgb]{0.56,0.35,0.01}{#1}}
\newcommand{\ControlFlowTok}[1]{\textcolor[rgb]{0.13,0.29,0.53}{\textbf{#1}}}
\newcommand{\DataTypeTok}[1]{\textcolor[rgb]{0.13,0.29,0.53}{#1}}
\newcommand{\DecValTok}[1]{\textcolor[rgb]{0.00,0.00,0.81}{#1}}
\newcommand{\DocumentationTok}[1]{\textcolor[rgb]{0.56,0.35,0.01}{\textbf{\textit{#1}}}}
\newcommand{\ErrorTok}[1]{\textcolor[rgb]{0.64,0.00,0.00}{\textbf{#1}}}
\newcommand{\ExtensionTok}[1]{#1}
\newcommand{\FloatTok}[1]{\textcolor[rgb]{0.00,0.00,0.81}{#1}}
\newcommand{\FunctionTok}[1]{\textcolor[rgb]{0.13,0.29,0.53}{\textbf{#1}}}
\newcommand{\ImportTok}[1]{#1}
\newcommand{\InformationTok}[1]{\textcolor[rgb]{0.56,0.35,0.01}{\textbf{\textit{#1}}}}
\newcommand{\KeywordTok}[1]{\textcolor[rgb]{0.13,0.29,0.53}{\textbf{#1}}}
\newcommand{\NormalTok}[1]{#1}
\newcommand{\OperatorTok}[1]{\textcolor[rgb]{0.81,0.36,0.00}{\textbf{#1}}}
\newcommand{\OtherTok}[1]{\textcolor[rgb]{0.56,0.35,0.01}{#1}}
\newcommand{\PreprocessorTok}[1]{\textcolor[rgb]{0.56,0.35,0.01}{\textit{#1}}}
\newcommand{\RegionMarkerTok}[1]{#1}
\newcommand{\SpecialCharTok}[1]{\textcolor[rgb]{0.81,0.36,0.00}{\textbf{#1}}}
\newcommand{\SpecialStringTok}[1]{\textcolor[rgb]{0.31,0.60,0.02}{#1}}
\newcommand{\StringTok}[1]{\textcolor[rgb]{0.31,0.60,0.02}{#1}}
\newcommand{\VariableTok}[1]{\textcolor[rgb]{0.00,0.00,0.00}{#1}}
\newcommand{\VerbatimStringTok}[1]{\textcolor[rgb]{0.31,0.60,0.02}{#1}}
\newcommand{\WarningTok}[1]{\textcolor[rgb]{0.56,0.35,0.01}{\textbf{\textit{#1}}}}
\usepackage{graphicx}
\makeatletter
\newsavebox\pandoc@box
\newcommand*\pandocbounded[1]{% scales image to fit in text height/width
  \sbox\pandoc@box{#1}%
  \Gscale@div\@tempa{\textheight}{\dimexpr\ht\pandoc@box+\dp\pandoc@box\relax}%
  \Gscale@div\@tempb{\linewidth}{\wd\pandoc@box}%
  \ifdim\@tempb\p@<\@tempa\p@\let\@tempa\@tempb\fi% select the smaller of both
  \ifdim\@tempa\p@<\p@\scalebox{\@tempa}{\usebox\pandoc@box}%
  \else\usebox{\pandoc@box}%
  \fi%
}
% Set default figure placement to htbp
\def\fps@figure{htbp}
\makeatother
\setlength{\emergencystretch}{3em} % prevent overfull lines
\providecommand{\tightlist}{%
  \setlength{\itemsep}{0pt}\setlength{\parskip}{0pt}}
\usepackage{bookmark}
\IfFileExists{xurl.sty}{\usepackage{xurl}}{} % add URL line breaks if available
\urlstyle{same}
\hypersetup{
  pdftitle={Tutorial 3 - Linear Mixed Models.Rmd},
  pdfauthor={SFR605},
  hidelinks,
  pdfcreator={LaTeX via pandoc}}

\title{Tutorial 3 - Linear Mixed Models.Rmd}
\author{SFR605}
\date{}

\begin{document}
\maketitle

For this tutorial, we will need to install and load some packages using
the \texttt{library} function.

\#Load libraries

\begin{Shaded}
\begin{Highlighting}[]
 \FunctionTok{library}\NormalTok{(here)}
\end{Highlighting}
\end{Shaded}

\begin{verbatim}
## Warning: package 'here' was built under R version 4.5.2
\end{verbatim}

\begin{verbatim}
## here() starts at C:/Users/sander.elliott/OneDrive - University of Maine System/Desktop/Github/SFR605
\end{verbatim}

\begin{Shaded}
\begin{Highlighting}[]
 \FunctionTok{library}\NormalTok{(ggplot2)}
\end{Highlighting}
\end{Shaded}

\begin{verbatim}
## Warning: package 'ggplot2' was built under R version 4.5.1
\end{verbatim}

\begin{Shaded}
\begin{Highlighting}[]
 \FunctionTok{library}\NormalTok{(lme4)}
\end{Highlighting}
\end{Shaded}

\begin{verbatim}
## Warning: package 'lme4' was built under R version 4.5.2
\end{verbatim}

\begin{verbatim}
## Loading required package: Matrix
\end{verbatim}

\begin{Shaded}
\begin{Highlighting}[]
 \FunctionTok{library}\NormalTok{(MuMIn)}
\end{Highlighting}
\end{Shaded}

\begin{verbatim}
## Warning: package 'MuMIn' was built under R version 4.5.2
\end{verbatim}

We are interested in whether fish trophic position increases with fish
size. To answer this question, we have measured body length in 3
different fish species with 10 individuals sampled per species across 6
different lakes. \emph{Lake = Lake ID }Fish\_Species = Fish species ID
\emph{Fish\_Length = length of the fish in mm }Trophic\_Pos = trophic
position {[}primary producers = 1, herbivores = 2, carnivores = 3 and
up{]}
\includegraphics[width=10.41667in,height=\textheight,keepaspectratio]{img/fish_trophic.jpg}
Let's read in the data and make some exploratory plots of the
relationship between trophic position and size.

\#Explore data

\begin{Shaded}
\begin{Highlighting}[]
\CommentTok{\# Load the dataset}
\NormalTok{fish.data }\OtherTok{\textless{}{-}} \FunctionTok{read.csv}\NormalTok{(}\FunctionTok{here}\NormalTok{(}\StringTok{\textquotesingle{}data/fishdata.csv\textquotesingle{}}\NormalTok{), }\AttributeTok{stringsAsFactors =} \ConstantTok{TRUE}\NormalTok{) }

\CommentTok{\# Set a simple theme for all ggplot figures}
\NormalTok{fig }\OtherTok{\textless{}{-}} \FunctionTok{theme\_bw}\NormalTok{() }\SpecialCharTok{+} 
  \FunctionTok{theme}\NormalTok{(}\AttributeTok{panel.grid.minor=}\FunctionTok{element\_blank}\NormalTok{(), }
        \AttributeTok{panel.grid.major=}\FunctionTok{element\_blank}\NormalTok{(), }
        \AttributeTok{panel.background=}\FunctionTok{element\_blank}\NormalTok{(), }
        \AttributeTok{strip.background=}\FunctionTok{element\_blank}\NormalTok{(), }
        \AttributeTok{strip.text.y =} \FunctionTok{element\_text}\NormalTok{(),}
        \AttributeTok{legend.background=}\FunctionTok{element\_blank}\NormalTok{(),}
        \AttributeTok{legend.key=}\FunctionTok{element\_blank}\NormalTok{(),}
        \AttributeTok{panel.border =} \FunctionTok{element\_rect}\NormalTok{(}\AttributeTok{colour=}\StringTok{"black"}\NormalTok{, }\AttributeTok{fill =} \ConstantTok{NA}\NormalTok{))}

\CommentTok{\# Basic plot aesthetics for the relationship we care about}
\NormalTok{plot }\OtherTok{\textless{}{-}} \FunctionTok{ggplot}\NormalTok{(}\FunctionTok{aes}\NormalTok{(Fish\_Length, Trophic\_Pos), }\AttributeTok{data =}\NormalTok{ fish.data)}

\CommentTok{\# Plot 1 {-} All data}
\NormalTok{plot }\SpecialCharTok{+} \FunctionTok{geom\_point}\NormalTok{() }\SpecialCharTok{+} 
  \FunctionTok{labs}\NormalTok{(}\AttributeTok{x =} \StringTok{"Length (mm)"}\NormalTok{, }\AttributeTok{y =} \StringTok{"Trophic position"}\NormalTok{, }
       \AttributeTok{title =} \StringTok{"All data"}\NormalTok{) }\SpecialCharTok{+} 
\NormalTok{  fig}
\end{Highlighting}
\end{Shaded}

\pandocbounded{\includegraphics[keepaspectratio]{03_Linear_Mixed_Models_files/figure-latex/unnamed-chunk-2-1.pdf}}

\begin{Shaded}
\begin{Highlighting}[]
\CommentTok{\# Plot 2 {-} By species}
\NormalTok{plot }\SpecialCharTok{+} \FunctionTok{geom\_point}\NormalTok{() }\SpecialCharTok{+} 
  \FunctionTok{facet\_wrap}\NormalTok{(}\SpecialCharTok{\textasciitilde{}}\NormalTok{ Fish\_Species) }\SpecialCharTok{+} 
  \FunctionTok{labs}\NormalTok{(}\AttributeTok{x =} \StringTok{"Length (mm)"}\NormalTok{, }\AttributeTok{y =} \StringTok{"Trophic position"}\NormalTok{, }
       \AttributeTok{title =} \StringTok{"By species"}\NormalTok{) }\SpecialCharTok{+} 
\NormalTok{  fig}
\end{Highlighting}
\end{Shaded}

\pandocbounded{\includegraphics[keepaspectratio]{03_Linear_Mixed_Models_files/figure-latex/unnamed-chunk-2-2.pdf}}

\begin{Shaded}
\begin{Highlighting}[]
\CommentTok{\# Plot 3 – By lake}
\NormalTok{plot }\SpecialCharTok{+} \FunctionTok{geom\_point}\NormalTok{() }\SpecialCharTok{+} 
  \FunctionTok{facet\_wrap}\NormalTok{(}\SpecialCharTok{\textasciitilde{}}\NormalTok{ Lake) }\SpecialCharTok{+} 
  \FunctionTok{labs}\NormalTok{(}\AttributeTok{x =} \StringTok{"Length (mm)"}\NormalTok{, }\AttributeTok{y =} \StringTok{"Trophic position"}\NormalTok{, }
       \AttributeTok{title =} \StringTok{"By lake"}\NormalTok{) }\SpecialCharTok{+} 
\NormalTok{  fig}
\end{Highlighting}
\end{Shaded}

\pandocbounded{\includegraphics[keepaspectratio]{03_Linear_Mixed_Models_files/figure-latex/unnamed-chunk-2-3.pdf}}

\#Question 1 Does the relationship between trophic position and length
appear to differ by species and/or lake? In what way(s)?

By species, It looks like the intercept changes, ascending order: 1,3,2.
The slope may also change. Species 1 looks like it may have a lower
slope than the other two. By lake it is a bit harder to tell, but it
does look like the intercept varies.

Now let's take a look at the data structure and the distribution of our
continuous variables. In this chunk, we will also re-scale our variables
to avoid convergence errors, and plot the residuals of a linear model
against our potential random effects. As you can see, the plots suggest
that there is residual variance that could be explained by these
factors, so they should be included in the model.

\#Explore data II

\begin{Shaded}
\begin{Highlighting}[]
\CommentTok{\# Look at data structure}
\FunctionTok{str}\NormalTok{(fish.data)}
\end{Highlighting}
\end{Shaded}

\begin{verbatim}
## 'data.frame':    180 obs. of  4 variables:
##  $ Lake        : Factor w/ 6 levels "L1","L2","L3",..: 1 1 1 1 1 1 1 1 1 1 ...
##  $ Fish_Species: Factor w/ 3 levels "S1","S2","S3": 1 1 1 1 1 1 1 1 1 1 ...
##  $ Fish_Length : num  105 195 294 414 237 ...
##  $ Trophic_Pos : num  2.6 2.7 2.74 2.74 2.79 ...
\end{verbatim}

\begin{Shaded}
\begin{Highlighting}[]
\CommentTok{\# Look at the distribution of samples for each factor}
\FunctionTok{table}\NormalTok{(fish.data[ , }\FunctionTok{c}\NormalTok{(}\StringTok{"Lake"}\NormalTok{, }\StringTok{"Fish\_Species"}\NormalTok{)])}
\end{Highlighting}
\end{Shaded}

\begin{verbatim}
##     Fish_Species
## Lake S1 S2 S3
##   L1 10 10 10
##   L2 10 10 10
##   L3 10 10 10
##   L4 10 10 10
##   L5 10 10 10
##   L6 10 10 10
\end{verbatim}

\begin{Shaded}
\begin{Highlighting}[]
\CommentTok{\# Look at the distribution of continuous variables:}
\FunctionTok{par}\NormalTok{(}\AttributeTok{mfrow =} \FunctionTok{c}\NormalTok{(}\DecValTok{1}\NormalTok{, }\DecValTok{2}\NormalTok{), }\AttributeTok{mar =} \FunctionTok{c}\NormalTok{(}\DecValTok{4}\NormalTok{, }\DecValTok{4}\NormalTok{, }\DecValTok{1}\NormalTok{, }\DecValTok{1}\NormalTok{))}
\FunctionTok{hist}\NormalTok{(fish.data}\SpecialCharTok{$}\NormalTok{Fish\_Length, }\AttributeTok{xlab =} \StringTok{"Length (mm)"}\NormalTok{, }\AttributeTok{main =} \StringTok{""}\NormalTok{)}
\FunctionTok{hist}\NormalTok{(fish.data}\SpecialCharTok{$}\NormalTok{Trophic\_Pos, }\AttributeTok{xlab =} \StringTok{"Trophic position"}\NormalTok{, }\AttributeTok{main =} \StringTok{""}\NormalTok{)}
\end{Highlighting}
\end{Shaded}

\pandocbounded{\includegraphics[keepaspectratio]{03_Linear_Mixed_Models_files/figure-latex/unnamed-chunk-3-1.pdf}}

\begin{Shaded}
\begin{Highlighting}[]
\CommentTok{\# Re{-}scale your variables}
\CommentTok{\# Standardized length, "by hand"}
\NormalTok{fish.data}\SpecialCharTok{$}\NormalTok{Z\_Length }\OtherTok{\textless{}{-}}\NormalTok{ (fish.data}\SpecialCharTok{$}\NormalTok{Fish\_Length }\SpecialCharTok{{-}} \FunctionTok{mean}\NormalTok{(fish.data}\SpecialCharTok{$}\NormalTok{Fish\_Length)) }\SpecialCharTok{/} 
                      \FunctionTok{sd}\NormalTok{(fish.data}\SpecialCharTok{$}\NormalTok{Fish\_Length)}

\CommentTok{\# Standardized trophic position, with the function scale}
\NormalTok{fish.data}\SpecialCharTok{$}\NormalTok{Z\_TP     }\OtherTok{\textless{}{-}} \FunctionTok{scale}\NormalTok{(fish.data}\SpecialCharTok{$}\NormalTok{Trophic\_Pos)}

\CommentTok{\# Simple linear test of relationship between fish length and trophic position}
\NormalTok{lm.test }\OtherTok{\textless{}{-}} \FunctionTok{lm}\NormalTok{(Z\_TP }\SpecialCharTok{\textasciitilde{}}\NormalTok{ Z\_Length, }\AttributeTok{data =}\NormalTok{ fish.data)}

\CommentTok{\#Save residuals}
\NormalTok{lm.test.resid }\OtherTok{\textless{}{-}} \FunctionTok{rstandard}\NormalTok{(lm.test)}

\CommentTok{\#Plot residuals by species and by lake}
\FunctionTok{par}\NormalTok{(}\AttributeTok{mfrow =} \FunctionTok{c}\NormalTok{(}\DecValTok{1}\NormalTok{, }\DecValTok{2}\NormalTok{))}

\FunctionTok{plot}\NormalTok{(lm.test.resid }\SpecialCharTok{\textasciitilde{}} \FunctionTok{as.factor}\NormalTok{(fish.data}\SpecialCharTok{$}\NormalTok{Fish\_Species),}
     \AttributeTok{xlab =} \StringTok{"Species"}\NormalTok{, }\AttributeTok{ylab =} \StringTok{"Standardized residuals"}\NormalTok{)}
\FunctionTok{abline}\NormalTok{(}\DecValTok{0}\NormalTok{, }\DecValTok{0}\NormalTok{, }\AttributeTok{lty =} \DecValTok{2}\NormalTok{)}

\FunctionTok{plot}\NormalTok{(lm.test.resid }\SpecialCharTok{\textasciitilde{}} \FunctionTok{as.factor}\NormalTok{(fish.data}\SpecialCharTok{$}\NormalTok{Lake),}
     \AttributeTok{xlab =} \StringTok{"Lake"}\NormalTok{, }\AttributeTok{ylab =} \StringTok{"Standardized residuals"}\NormalTok{)}
\FunctionTok{abline}\NormalTok{(}\DecValTok{0}\NormalTok{, }\DecValTok{0}\NormalTok{, }\AttributeTok{lty =} \DecValTok{2}\NormalTok{)}
\end{Highlighting}
\end{Shaded}

\pandocbounded{\includegraphics[keepaspectratio]{03_Linear_Mixed_Models_files/figure-latex/unnamed-chunk-3-2.pdf}}

\begin{Shaded}
\begin{Highlighting}[]
\CommentTok{\#Fit a linear mixed model with lake and fish species as random effects}
\FunctionTok{lmer}\NormalTok{(Z\_TP }\SpecialCharTok{\textasciitilde{}}\NormalTok{ Z\_Length }\SpecialCharTok{+}\NormalTok{ (}\DecValTok{1} \SpecialCharTok{|}\NormalTok{ Lake) }\SpecialCharTok{+}\NormalTok{ (}\DecValTok{1} \SpecialCharTok{|}\NormalTok{ Fish\_Species),}
     \AttributeTok{data =}\NormalTok{ fish.data, }\AttributeTok{REML =} \ConstantTok{TRUE}\NormalTok{)}
\end{Highlighting}
\end{Shaded}

\begin{verbatim}
## Linear mixed model fit by REML ['lmerMod']
## Formula: Z_TP ~ Z_Length + (1 | Lake) + (1 | Fish_Species)
##    Data: fish.data
## REML criterion at convergence: 72.4662
## Random effects:
##  Groups       Name        Std.Dev.
##  Lake         (Intercept) 0.4516  
##  Fish_Species (Intercept) 0.9301  
##  Residual                 0.2605  
## Number of obs: 180, groups:  Lake, 6; Fish_Species, 3
## Fixed Effects:
## (Intercept)     Z_Length  
##   9.752e-14    4.198e-01
\end{verbatim}

\#Question 2 Now develop a linear mixed model that allows the slopes of
the relationship between trophic position and body length to vary by
lake and species. Note: You don't have to evaluate how well the model
works or fix any warnings, just show me how you would code it.

\begin{Shaded}
\begin{Highlighting}[]
\FunctionTok{lmer}\NormalTok{(Z\_TP }\SpecialCharTok{\textasciitilde{}}\NormalTok{ Z\_Length }\SpecialCharTok{+}\NormalTok{ (}\DecValTok{1} \SpecialCharTok{+}\NormalTok{ Z\_Length }\SpecialCharTok{|}\NormalTok{ Lake) }\SpecialCharTok{+}\NormalTok{ (}\DecValTok{1} \SpecialCharTok{+}\NormalTok{ Z\_Length }\SpecialCharTok{|}\NormalTok{ Fish\_Species),}
     \AttributeTok{data =}\NormalTok{ fish.data, }\AttributeTok{REML =} \ConstantTok{TRUE}\NormalTok{)}
\end{Highlighting}
\end{Shaded}

\begin{verbatim}
## boundary (singular) fit: see help('isSingular')
\end{verbatim}

\begin{verbatim}
## Linear mixed model fit by REML ['lmerMod']
## Formula: 
## Z_TP ~ Z_Length + (1 + Z_Length | Lake) + (1 + Z_Length | Fish_Species)
##    Data: fish.data
## REML criterion at convergence: 20.5786
## Random effects:
##  Groups       Name        Std.Dev. Corr 
##  Lake         (Intercept) 0.45279       
##               Z_Length    0.02378  -0.82
##  Fish_Species (Intercept) 0.93103       
##               Z_Length    0.15728  1.00 
##  Residual                 0.22341       
## Number of obs: 180, groups:  Lake, 6; Fish_Species, 3
## Fixed Effects:
## (Intercept)     Z_Length  
##  -0.0009025    0.4223738  
## optimizer (nloptwrap) convergence code: 0 (OK) ; 0 optimizer warnings; 1 lme4 warnings
\end{verbatim}

\#Question 3 Next let's make a list of 7 different mixed model
structures that could be compared to a basic linear model. Remember to
set REML = FALSE in order to compare your new models with the basic
linear model where estimation method = ML. Use the naming convention M1
- M8.

\begin{Shaded}
\begin{Highlighting}[]
\CommentTok{\#Linear model}
\NormalTok{M0 }\OtherTok{\textless{}{-}} \FunctionTok{lm}\NormalTok{(Z\_TP }\SpecialCharTok{\textasciitilde{}}\NormalTok{ Z\_Length, }\AttributeTok{data =}\NormalTok{ fish.data)}
\CommentTok{\# Full model with varying intercepts \#Hint: you already defined this above}
\NormalTok{M1 }\OtherTok{\textless{}{-}} \FunctionTok{lmer}\NormalTok{(Z\_TP }\SpecialCharTok{\textasciitilde{}}\NormalTok{ Z\_Length }\SpecialCharTok{+}\NormalTok{ (}\DecValTok{1} \SpecialCharTok{|}\NormalTok{ Lake) }\SpecialCharTok{+}\NormalTok{ (}\DecValTok{1} \SpecialCharTok{|}\NormalTok{ Fish\_Species),}
     \AttributeTok{data =}\NormalTok{ fish.data, }\AttributeTok{REML =} \ConstantTok{FALSE}\NormalTok{)}
\CommentTok{\# Full model with varying intercepts and slopes \#Hint: you already defined this above}
\NormalTok{M2 }\OtherTok{\textless{}{-}} \FunctionTok{lmer}\NormalTok{(Z\_TP }\SpecialCharTok{\textasciitilde{}}\NormalTok{ Z\_Length }\SpecialCharTok{+}\NormalTok{ (}\DecValTok{1} \SpecialCharTok{+}\NormalTok{ Z\_Length }\SpecialCharTok{|}\NormalTok{ Lake) }\SpecialCharTok{+}\NormalTok{ (}\DecValTok{1} \SpecialCharTok{+}\NormalTok{ Z\_Length }\SpecialCharTok{|}\NormalTok{ Fish\_Species),}
     \AttributeTok{data =}\NormalTok{ fish.data, }\AttributeTok{REML =} \ConstantTok{FALSE}\NormalTok{)}
\end{Highlighting}
\end{Shaded}

\begin{verbatim}
## boundary (singular) fit: see help('isSingular')
\end{verbatim}

\begin{Shaded}
\begin{Highlighting}[]
\CommentTok{\# No Lake, varying intercepts only}
\NormalTok{M3 }\OtherTok{\textless{}{-}} \FunctionTok{lmer}\NormalTok{(Z\_TP }\SpecialCharTok{\textasciitilde{}}\NormalTok{ Z\_Length }\SpecialCharTok{+}\NormalTok{ (}\DecValTok{1} \SpecialCharTok{|}\NormalTok{ Fish\_Species),}
           \AttributeTok{data =}\NormalTok{ fish.data, }\AttributeTok{REML =} \ConstantTok{FALSE}\NormalTok{)}
\CommentTok{\# No Species, varying intercepts only}
\NormalTok{M4 }\OtherTok{\textless{}{-}} \FunctionTok{lmer}\NormalTok{(Z\_TP }\SpecialCharTok{\textasciitilde{}}\NormalTok{ Z\_Length }\SpecialCharTok{+}\NormalTok{ (}\DecValTok{1} \SpecialCharTok{|}\NormalTok{ Lake),}
           \AttributeTok{data =}\NormalTok{ fish.data, }\AttributeTok{REML =} \ConstantTok{FALSE}\NormalTok{)}
\CommentTok{\# No Lake, varying intercepts and slopes}
\NormalTok{M5 }\OtherTok{\textless{}{-}} \FunctionTok{lmer}\NormalTok{(Z\_TP }\SpecialCharTok{\textasciitilde{}}\NormalTok{ Z\_Length }\SpecialCharTok{+}\NormalTok{ (}\DecValTok{1} \SpecialCharTok{+}\NormalTok{ Z\_Length }\SpecialCharTok{|}\NormalTok{ Fish\_Species),}
     \AttributeTok{data =}\NormalTok{ fish.data, }\AttributeTok{REML =} \ConstantTok{FALSE}\NormalTok{)}
\end{Highlighting}
\end{Shaded}

\begin{verbatim}
## boundary (singular) fit: see help('isSingular')
\end{verbatim}

\begin{Shaded}
\begin{Highlighting}[]
\CommentTok{\# No Species, varying intercepts and slopes}
\NormalTok{M6 }\OtherTok{\textless{}{-}} \FunctionTok{lmer}\NormalTok{(Z\_TP }\SpecialCharTok{\textasciitilde{}}\NormalTok{ Z\_Length }\SpecialCharTok{+}\NormalTok{ (}\DecValTok{1} \SpecialCharTok{+}\NormalTok{ Z\_Length }\SpecialCharTok{|}\NormalTok{ Lake),}
     \AttributeTok{data =}\NormalTok{ fish.data, }\AttributeTok{REML =} \ConstantTok{FALSE}\NormalTok{)}
\end{Highlighting}
\end{Shaded}

\begin{verbatim}
## boundary (singular) fit: see help('isSingular')
\end{verbatim}

\begin{Shaded}
\begin{Highlighting}[]
\CommentTok{\# Full model with varying intercepts, and slopes only varying by lake}
\NormalTok{M7 }\OtherTok{\textless{}{-}} \FunctionTok{lmer}\NormalTok{(Z\_TP }\SpecialCharTok{\textasciitilde{}}\NormalTok{ Z\_Length }\SpecialCharTok{+}\NormalTok{ (}\DecValTok{1} \SpecialCharTok{+}\NormalTok{ Z\_Length }\SpecialCharTok{|}\NormalTok{ Lake) }\SpecialCharTok{+}\NormalTok{ (}\DecValTok{1} \SpecialCharTok{|}\NormalTok{ Fish\_Species),}
     \AttributeTok{data =}\NormalTok{ fish.data, }\AttributeTok{REML =} \ConstantTok{FALSE}\NormalTok{)}
\end{Highlighting}
\end{Shaded}

\begin{verbatim}
## boundary (singular) fit: see help('isSingular')
\end{verbatim}

\begin{Shaded}
\begin{Highlighting}[]
\CommentTok{\# Full model with varying intercepts and slopes only varying by species}
\NormalTok{M8 }\OtherTok{\textless{}{-}} \FunctionTok{lmer}\NormalTok{(Z\_TP }\SpecialCharTok{\textasciitilde{}}\NormalTok{ Z\_Length }\SpecialCharTok{+}\NormalTok{ (}\DecValTok{1} \SpecialCharTok{+}\NormalTok{ Z\_Length }\SpecialCharTok{|}\NormalTok{ Fish\_Species) }\SpecialCharTok{+}\NormalTok{ (}\DecValTok{1} \SpecialCharTok{|}\NormalTok{ Lake),}
           \AttributeTok{data =}\NormalTok{ fish.data, }\AttributeTok{REML =} \ConstantTok{FALSE}\NormalTok{)}
\end{Highlighting}
\end{Shaded}

\begin{verbatim}
## boundary (singular) fit: see help('isSingular')
\end{verbatim}

Now that you have defined your models, lets estimate the AICc value for
the first model using the package MuMin.

\#Compare models

\begin{Shaded}
\begin{Highlighting}[]
\NormalTok{MuMIn}\SpecialCharTok{::}\FunctionTok{AICc}\NormalTok{(M0)}
\end{Highlighting}
\end{Shaded}

\begin{verbatim}
## [1] 479.8591
\end{verbatim}

\begin{Shaded}
\begin{Highlighting}[]
\CommentTok{\# To group all AICc values into a single table, we can use MuMIn::model.sel() to calculate AICc for each model (along with other outputs)}
\NormalTok{AIC.table  }\OtherTok{\textless{}{-}}\NormalTok{ MuMIn}\SpecialCharTok{::}\FunctionTok{model.sel}\NormalTok{(M0, M1, M2, M3, M4, M5, M6, M7, M8)}

\CommentTok{\# Then we can select only the columns of interest to print into a table }
\CommentTok{\# \textasciigrave{}df\textasciigrave{} is the degree of freedom}
\CommentTok{\# \textasciigrave{}logLik\textasciigrave{} is the loglikelihood}
\CommentTok{\# \textasciigrave{}delta\textasciigrave{} is the AICc difference with the lowest value}
\NormalTok{(AIC.table }\OtherTok{\textless{}{-}}\NormalTok{ AIC.table[ , }\FunctionTok{c}\NormalTok{(}\StringTok{"df"}\NormalTok{, }\StringTok{"logLik"}\NormalTok{, }\StringTok{"AICc"}\NormalTok{, }\StringTok{"delta"}\NormalTok{)])}
\end{Highlighting}
\end{Shaded}

\begin{verbatim}
##    df      logLik      AICc      delta
## M8  7   -8.597929  31.84702   0.000000
## M2  9   -8.216019  35.49086   3.643839
## M1  5  -33.480080  77.30499  45.457965
## M7  7  -33.186374  81.02391  49.176890
## M5  6 -128.310995 269.10754 237.260517
## M3  4 -134.532965 277.29450 245.447480
## M4  4 -224.715763 457.66010 425.813076
## M6  6 -224.671201 461.82795 429.980930
## M0  3 -236.861362 479.85909 448.012065
\end{verbatim}

\begin{Shaded}
\begin{Highlighting}[]
\CommentTok{\# For more information on the other outputs/results returned by the function \textasciigrave{}model.sel()\textasciigrave{}, see \textasciigrave{}?model.sel\textasciigrave{}.}
\end{Highlighting}
\end{Shaded}

Let's take a closer look at M8 and M2. Hint: If these did not have the
lowest AICc values, check your model construction. Because we are
comparing two mixed effect models, we can set \texttt{REML\ =\ TRUE}
when re-defining M8 and M2.

\#Question 4 Re-define M8 and M2 so that \texttt{REML\ =\ TRUE} and
print a table to compare the two models, similar to the chunk above.

\begin{Shaded}
\begin{Highlighting}[]
\CommentTok{\#REML}
\NormalTok{M2 }\OtherTok{\textless{}{-}} \FunctionTok{lmer}\NormalTok{(Z\_TP }\SpecialCharTok{\textasciitilde{}}\NormalTok{ Z\_Length }\SpecialCharTok{+}\NormalTok{ (}\DecValTok{1} \SpecialCharTok{+}\NormalTok{ Z\_Length }\SpecialCharTok{|}\NormalTok{ Lake) }\SpecialCharTok{+}\NormalTok{ (}\DecValTok{1} \SpecialCharTok{+}\NormalTok{ Z\_Length }\SpecialCharTok{|}\NormalTok{ Fish\_Species),}
     \AttributeTok{data =}\NormalTok{ fish.data, }\AttributeTok{REML =} \ConstantTok{TRUE}\NormalTok{)}
\end{Highlighting}
\end{Shaded}

\begin{verbatim}
## boundary (singular) fit: see help('isSingular')
\end{verbatim}

\begin{Shaded}
\begin{Highlighting}[]
\NormalTok{M8 }\OtherTok{\textless{}{-}} \FunctionTok{lmer}\NormalTok{(Z\_TP }\SpecialCharTok{\textasciitilde{}}\NormalTok{ Z\_Length }\SpecialCharTok{+}\NormalTok{ (}\DecValTok{1} \SpecialCharTok{+}\NormalTok{ Z\_Length }\SpecialCharTok{|}\NormalTok{ Fish\_Species) }\SpecialCharTok{+}\NormalTok{ (}\DecValTok{1} \SpecialCharTok{|}\NormalTok{ Lake),}
           \AttributeTok{data =}\NormalTok{ fish.data, }\AttributeTok{REML =} \ConstantTok{TRUE}\NormalTok{)}
\end{Highlighting}
\end{Shaded}

\begin{verbatim}
## boundary (singular) fit: see help('isSingular')
\end{verbatim}

\begin{Shaded}
\begin{Highlighting}[]
\NormalTok{AIC.table  }\OtherTok{\textless{}{-}}\NormalTok{ MuMIn}\SpecialCharTok{::}\FunctionTok{model.sel}\NormalTok{(M2, M8)}

\NormalTok{(AIC.table }\OtherTok{\textless{}{-}}\NormalTok{ AIC.table[ , }\FunctionTok{c}\NormalTok{(}\StringTok{"df"}\NormalTok{, }\StringTok{"logLik"}\NormalTok{, }\StringTok{"AICc"}\NormalTok{, }\StringTok{"delta"}\NormalTok{)])}
\end{Highlighting}
\end{Shaded}

\begin{verbatim}
##    df    logLik     AICc    delta
## M8  7 -10.84011 36.33137 0.000000
## M2  9 -10.28932 39.63747 3.306098
\end{verbatim}

Keeping \texttt{REML=TRUE}, let's plot predicted vs residual values of
M8.

\#Plot residuals

\begin{Shaded}
\begin{Highlighting}[]
\CommentTok{\# Plot predicted values vs residual values}
\FunctionTok{par}\NormalTok{(}\AttributeTok{mar=}\FunctionTok{c}\NormalTok{(}\DecValTok{4}\NormalTok{,}\DecValTok{4}\NormalTok{,.}\DecValTok{5}\NormalTok{,.}\DecValTok{5}\NormalTok{))}
\FunctionTok{plot}\NormalTok{(}\FunctionTok{resid}\NormalTok{(M8) }\SpecialCharTok{\textasciitilde{}} \FunctionTok{fitted}\NormalTok{(M8), }
     \AttributeTok{xlab =} \StringTok{\textquotesingle{}Predicted values\textquotesingle{}}\NormalTok{, }
     \AttributeTok{ylab =} \StringTok{\textquotesingle{}Normalized residuals\textquotesingle{}}\NormalTok{)}
\FunctionTok{abline}\NormalTok{(}\AttributeTok{h =} \DecValTok{0}\NormalTok{, }\AttributeTok{lty =} \DecValTok{2}\NormalTok{)}
\end{Highlighting}
\end{Shaded}

\pandocbounded{\includegraphics[keepaspectratio]{03_Linear_Mixed_Models_files/figure-latex/unnamed-chunk-8-1.pdf}}

\begin{Shaded}
\begin{Highlighting}[]
\CommentTok{\# Homogeneous dispersion of the residuals means that the assumption is respected.}

\CommentTok{\# In order to check the independence of the model residuals we need to plot residuals vs each covariate of the model}
\FunctionTok{par}\NormalTok{(}\AttributeTok{mfrow =} \FunctionTok{c}\NormalTok{(}\DecValTok{1}\NormalTok{,}\DecValTok{3}\NormalTok{), }\AttributeTok{mar=}\FunctionTok{c}\NormalTok{(}\DecValTok{4}\NormalTok{,}\DecValTok{4}\NormalTok{,.}\DecValTok{5}\NormalTok{,.}\DecValTok{5}\NormalTok{))}

\FunctionTok{plot}\NormalTok{(}\FunctionTok{resid}\NormalTok{(M8) }\SpecialCharTok{\textasciitilde{}}\NormalTok{ fish.data}\SpecialCharTok{$}\NormalTok{Z\_Length, }
     \AttributeTok{xlab =} \StringTok{"Length"}\NormalTok{, }\AttributeTok{ylab =} \StringTok{"Normalized residuals"}\NormalTok{)}
\FunctionTok{abline}\NormalTok{(}\AttributeTok{h =} \DecValTok{0}\NormalTok{, }\AttributeTok{lty =} \DecValTok{2}\NormalTok{)}

\FunctionTok{boxplot}\NormalTok{(}\FunctionTok{resid}\NormalTok{(M8) }\SpecialCharTok{\textasciitilde{}}\NormalTok{ Fish\_Species, }\AttributeTok{data =}\NormalTok{ fish.data, }
        \AttributeTok{xlab =} \StringTok{"Species"}\NormalTok{, }\AttributeTok{ylab =} \StringTok{"Normalized residuals"}\NormalTok{)}
\FunctionTok{abline}\NormalTok{(}\AttributeTok{h =} \DecValTok{0}\NormalTok{, }\AttributeTok{lty =} \DecValTok{2}\NormalTok{)}

\FunctionTok{boxplot}\NormalTok{(}\FunctionTok{resid}\NormalTok{(M8) }\SpecialCharTok{\textasciitilde{}}\NormalTok{ Lake, }\AttributeTok{data =}\NormalTok{ fish.data, }
        \AttributeTok{xlab =} \StringTok{"Lakes"}\NormalTok{, }\AttributeTok{ylab =} \StringTok{"Normalized residuals"}\NormalTok{)}
\FunctionTok{abline}\NormalTok{(}\AttributeTok{h =} \DecValTok{0}\NormalTok{, }\AttributeTok{lty =} \DecValTok{2}\NormalTok{)}
\end{Highlighting}
\end{Shaded}

\pandocbounded{\includegraphics[keepaspectratio]{03_Linear_Mixed_Models_files/figure-latex/unnamed-chunk-8-2.pdf}}

\begin{Shaded}
\begin{Highlighting}[]
\CommentTok{\# Homogeneous dispersion of the residuals around 0 means no pattern of residuals depending on the variable, therefore the assumption is respected!}
\CommentTok{\# Note: The clusters are due to the data structure, where fish of only 5 size classes (large, small, and three groups in between) were captured.}

\CommentTok{\# Check the normality of the model residuals as residuals following a normal distribution indicate that the model is not biased.}
\FunctionTok{hist}\NormalTok{(}\FunctionTok{resid}\NormalTok{(M8))}
\CommentTok{\# The residuals are normal! This means our model is not biased.}

\CommentTok{\# Now we are ready for interpretation and visualization}
\CommentTok{\# Let\textquotesingle{}s take a closer look at our final model using the \textasciigrave{}summary()\textasciigrave{} function. }
\NormalTok{(summ\_M8 }\OtherTok{\textless{}{-}} \FunctionTok{summary}\NormalTok{(M8))}
\end{Highlighting}
\end{Shaded}

\begin{verbatim}
## Linear mixed model fit by REML ['lmerMod']
## Formula: Z_TP ~ Z_Length + (1 + Z_Length | Fish_Species) + (1 | Lake)
##    Data: fish.data
## 
## REML criterion at convergence: 21.7
## 
## Scaled residuals: 
##      Min       1Q   Median       3Q      Max 
## -2.77186 -0.60166  0.05589  0.64239  2.27775 
## 
## Random effects:
##  Groups       Name        Variance Std.Dev. Corr
##  Lake         (Intercept) 0.20502  0.4528       
##  Fish_Species (Intercept) 0.86707  0.9312       
##               Z_Length    0.02466  0.1570   1.00
##  Residual                 0.05039  0.2245       
## Number of obs: 180, groups:  Lake, 6; Fish_Species, 3
## 
## Fixed effects:
##               Estimate Std. Error t value
## (Intercept) -0.0009059  0.5687462  -0.002
## Z_Length     0.4222697  0.0922075   4.580
## 
## Correlation of Fixed Effects:
##          (Intr)
## Z_Length 0.929 
## optimizer (nloptwrap) convergence code: 0 (OK)
## boundary (singular) fit: see help('isSingular')
\end{verbatim}

\pandocbounded{\includegraphics[keepaspectratio]{03_Linear_Mixed_Models_files/figure-latex/unnamed-chunk-8-3.pdf}}

\#Question 5 What is the slope and confidence interval of the Z\_Length
variable in the M8 model? Slope = 0.4222697. Confidence interval =
0.241543 - 0.6029964

Is the slope of Z\_Length significantly different from 0? Yes, 0 is not
inside the confidence interval.

Remember from the lecture that you can calculate the 95\% confidence
interval (CI) with this equation:
\texttt{CI\ =\ Estimate\ ±\ 1.96\ ∗\ Std.\ Error} If 0 is in the
interval, then the parameter is not significantly different from zero at
a threshold of α = 0.05.

\begin{Shaded}
\begin{Highlighting}[]
\NormalTok{CIl }\OtherTok{\textless{}{-}} \FloatTok{0.4222697} \SpecialCharTok{{-}} \FloatTok{1.96} \SpecialCharTok{*} \FloatTok{0.0922075}
\NormalTok{CIu }\OtherTok{\textless{}{-}} \FloatTok{0.4222697} \SpecialCharTok{+} \FloatTok{1.96} \SpecialCharTok{*} \FloatTok{0.0922075}
\NormalTok{CIl}
\end{Highlighting}
\end{Shaded}

\begin{verbatim}
## [1] 0.241543
\end{verbatim}

\begin{Shaded}
\begin{Highlighting}[]
\NormalTok{CIu}
\end{Highlighting}
\end{Shaded}

\begin{verbatim}
## [1] 0.6029964
\end{verbatim}

\#Question 6 Make two plots to graphically visualize the different
intercepts and slopes of the model to better interpret the results.
Hint: You will need to extract the coefficients of the full model (from
the model summary) and the coefficients of each level of the model,
which can be obtained with the \texttt{coef} function.

Color your data points by species in plot 1 and by lake in plot 2. Plot
a regression line for each species and lake in plots 1 and 2,
respectively. Hint: If using ggplot, you can add lines using
\texttt{geom\_abline} and define the intercept and slope from a table of
your extracted coefficients.

\begin{Shaded}
\begin{Highlighting}[]
\CommentTok{\# Model 8 Lake}
\FunctionTok{library}\NormalTok{(tidyverse)}
\end{Highlighting}
\end{Shaded}

\begin{verbatim}
## Warning: package 'tidyverse' was built under R version 4.5.2
\end{verbatim}

\begin{verbatim}
## Warning: package 'tidyr' was built under R version 4.5.1
\end{verbatim}

\begin{verbatim}
## Warning: package 'readr' was built under R version 4.5.1
\end{verbatim}

\begin{verbatim}
## Warning: package 'purrr' was built under R version 4.5.1
\end{verbatim}

\begin{verbatim}
## Warning: package 'stringr' was built under R version 4.5.1
\end{verbatim}

\begin{verbatim}
## Warning: package 'forcats' was built under R version 4.5.2
\end{verbatim}

\begin{verbatim}
## Warning: package 'lubridate' was built under R version 4.5.1
\end{verbatim}

\begin{verbatim}
## -- Attaching core tidyverse packages ------------------------ tidyverse 2.0.0 --
## v dplyr     1.1.4     v readr     2.1.5
## v forcats   1.0.1     v stringr   1.5.1
## v lubridate 1.9.4     v tibble    3.2.1
## v purrr     1.1.0     v tidyr     1.3.1
## -- Conflicts ------------------------------------------ tidyverse_conflicts() --
## x tidyr::expand() masks Matrix::expand()
## x dplyr::filter() masks stats::filter()
## x dplyr::lag()    masks stats::lag()
## x tidyr::pack()   masks Matrix::pack()
## x tidyr::unpack() masks Matrix::unpack()
## i Use the conflicted package (<http://conflicted.r-lib.org/>) to force all conflicts to become errors
\end{verbatim}

\begin{Shaded}
\begin{Highlighting}[]
\NormalTok{c }\OtherTok{\textless{}{-}} \FunctionTok{coef}\NormalTok{(M8)}
\NormalTok{c}
\end{Highlighting}
\end{Shaded}

\begin{verbatim}
## $Lake
##     (Intercept)  Z_Length
## L1 -0.085984021 0.4222697
## L2  0.002205203 0.4222697
## L3 -0.301816374 0.4222697
## L4 -0.574039393 0.4222697
## L5  0.218650002 0.4222697
## L6  0.735549184 0.4222697
## 
## $Fish_Species
##    (Intercept)  Z_Length
## S1  -1.0752983 0.2410741
## S2   0.5597870 0.5168302
## S3   0.5127937 0.5089048
## 
## attr(,"class")
## [1] "coef.mer"
\end{verbatim}

\begin{Shaded}
\begin{Highlighting}[]
\NormalTok{d }\OtherTok{\textless{}{-}}\NormalTok{ c}\SpecialCharTok{$}\NormalTok{Lake }\SpecialCharTok{\%\textgreater{}\%} 
\NormalTok{  tibble}\SpecialCharTok{::}\FunctionTok{rownames\_to\_column}\NormalTok{(}\StringTok{"Lake"}\NormalTok{)}
\NormalTok{d}
\end{Highlighting}
\end{Shaded}

\begin{verbatim}
##   Lake  (Intercept)  Z_Length
## 1   L1 -0.085984021 0.4222697
## 2   L2  0.002205203 0.4222697
## 3   L3 -0.301816374 0.4222697
## 4   L4 -0.574039393 0.4222697
## 5   L5  0.218650002 0.4222697
## 6   L6  0.735549184 0.4222697
\end{verbatim}

\begin{Shaded}
\begin{Highlighting}[]
\FunctionTok{head}\NormalTok{(d)}
\end{Highlighting}
\end{Shaded}

\begin{verbatim}
##   Lake  (Intercept)  Z_Length
## 1   L1 -0.085984021 0.4222697
## 2   L2  0.002205203 0.4222697
## 3   L3 -0.301816374 0.4222697
## 4   L4 -0.574039393 0.4222697
## 5   L5  0.218650002 0.4222697
## 6   L6  0.735549184 0.4222697
\end{verbatim}

\begin{Shaded}
\begin{Highlighting}[]
\FunctionTok{head}\NormalTok{(fish.data)}
\end{Highlighting}
\end{Shaded}

\begin{verbatim}
##   Lake Fish_Species Fish_Length Trophic_Pos   Z_Length      Z_TP
## 1   L1           S1    105.1501    2.602388 -1.5050527 -1.493034
## 2   L1           S1    194.5708    2.703522 -0.6180335 -1.287470
## 3   L1           S1    294.3636    2.742878  0.3718739 -1.207476
## 4   L1           S1    413.5295    2.737743  1.5539556 -1.217914
## 5   L1           S1    237.4739    2.785936 -0.1924502 -1.119959
## 6   L1           S1    107.9315    2.723862 -1.4774623 -1.246128
\end{verbatim}

\begin{Shaded}
\begin{Highlighting}[]
\FunctionTok{ggplot}\NormalTok{(}\AttributeTok{data =}\NormalTok{ fish.data, }\FunctionTok{aes}\NormalTok{(}\AttributeTok{x =}\NormalTok{ Z\_Length, }\AttributeTok{y =}\NormalTok{ Z\_TP, }\AttributeTok{colour =}\NormalTok{ Lake)) }\SpecialCharTok{+}
  \FunctionTok{geom\_point}\NormalTok{() }\SpecialCharTok{+}
  \FunctionTok{geom\_abline}\NormalTok{(}\AttributeTok{data =}\NormalTok{ d,}
              \FunctionTok{aes}\NormalTok{(}\AttributeTok{intercept =} \StringTok{\textasciigrave{}}\AttributeTok{(Intercept)}\StringTok{\textasciigrave{}}\NormalTok{, }\AttributeTok{slope =}\NormalTok{ Z\_Length, }\AttributeTok{colour =}\NormalTok{ Lake )) }\SpecialCharTok{+}
  \FunctionTok{theme\_classic}\NormalTok{()}
\end{Highlighting}
\end{Shaded}

\pandocbounded{\includegraphics[keepaspectratio]{03_Linear_Mixed_Models_files/figure-latex/unnamed-chunk-10-1.pdf}}

\begin{Shaded}
\begin{Highlighting}[]
\CommentTok{\# Model 8 Species}

\NormalTok{c }\OtherTok{\textless{}{-}} \FunctionTok{coef}\NormalTok{(M8)}
\NormalTok{c}
\end{Highlighting}
\end{Shaded}

\begin{verbatim}
## $Lake
##     (Intercept)  Z_Length
## L1 -0.085984021 0.4222697
## L2  0.002205203 0.4222697
## L3 -0.301816374 0.4222697
## L4 -0.574039393 0.4222697
## L5  0.218650002 0.4222697
## L6  0.735549184 0.4222697
## 
## $Fish_Species
##    (Intercept)  Z_Length
## S1  -1.0752983 0.2410741
## S2   0.5597870 0.5168302
## S3   0.5127937 0.5089048
## 
## attr(,"class")
## [1] "coef.mer"
\end{verbatim}

\begin{Shaded}
\begin{Highlighting}[]
\NormalTok{d }\OtherTok{\textless{}{-}}\NormalTok{ c}\SpecialCharTok{$}\NormalTok{Fish\_Species }\SpecialCharTok{\%\textgreater{}\%} 
\NormalTok{  tibble}\SpecialCharTok{::}\FunctionTok{rownames\_to\_column}\NormalTok{(}\StringTok{"Fish\_Species"}\NormalTok{)}
\NormalTok{d}
\end{Highlighting}
\end{Shaded}

\begin{verbatim}
##   Fish_Species (Intercept)  Z_Length
## 1           S1  -1.0752983 0.2410741
## 2           S2   0.5597870 0.5168302
## 3           S3   0.5127937 0.5089048
\end{verbatim}

\begin{Shaded}
\begin{Highlighting}[]
\FunctionTok{head}\NormalTok{(d)}
\end{Highlighting}
\end{Shaded}

\begin{verbatim}
##   Fish_Species (Intercept)  Z_Length
## 1           S1  -1.0752983 0.2410741
## 2           S2   0.5597870 0.5168302
## 3           S3   0.5127937 0.5089048
\end{verbatim}

\begin{Shaded}
\begin{Highlighting}[]
\FunctionTok{head}\NormalTok{(fish.data)}
\end{Highlighting}
\end{Shaded}

\begin{verbatim}
##   Lake Fish_Species Fish_Length Trophic_Pos   Z_Length      Z_TP
## 1   L1           S1    105.1501    2.602388 -1.5050527 -1.493034
## 2   L1           S1    194.5708    2.703522 -0.6180335 -1.287470
## 3   L1           S1    294.3636    2.742878  0.3718739 -1.207476
## 4   L1           S1    413.5295    2.737743  1.5539556 -1.217914
## 5   L1           S1    237.4739    2.785936 -0.1924502 -1.119959
## 6   L1           S1    107.9315    2.723862 -1.4774623 -1.246128
\end{verbatim}

\begin{Shaded}
\begin{Highlighting}[]
\FunctionTok{ggplot}\NormalTok{(}\AttributeTok{data =}\NormalTok{ fish.data, }\FunctionTok{aes}\NormalTok{(}\AttributeTok{x =}\NormalTok{ Z\_Length, }\AttributeTok{y =}\NormalTok{ Z\_TP, }\AttributeTok{colour =}\NormalTok{ Fish\_Species)) }\SpecialCharTok{+}
  \FunctionTok{geom\_point}\NormalTok{() }\SpecialCharTok{+}
  \FunctionTok{geom\_abline}\NormalTok{(}\AttributeTok{data =}\NormalTok{ d,}
              \FunctionTok{aes}\NormalTok{(}\AttributeTok{intercept =} \StringTok{\textasciigrave{}}\AttributeTok{(Intercept)}\StringTok{\textasciigrave{}}\NormalTok{, }\AttributeTok{slope =}\NormalTok{ Z\_Length, }\AttributeTok{colour =}\NormalTok{ Fish\_Species )) }\SpecialCharTok{+}
  \FunctionTok{theme\_classic}\NormalTok{()}
\end{Highlighting}
\end{Shaded}

\pandocbounded{\includegraphics[keepaspectratio]{03_Linear_Mixed_Models_files/figure-latex/unnamed-chunk-11-1.pdf}}

\#Acknowledgements and Copyright Information

\begin{verbatim}
Parts of this tutorial are adapted from a workshop originally developed by Catherine Baltazar, Dalal Hanna, Jacob Ziegler, Eric Pedersen and Zofia Taranu, and some parts originally revised in French by Cédric Frenette Dussault. 
https://r.qcbs.ca/workshops/r-workshop-07/ 
The workshop was modified for SFR 605 according to its license: CC BY-NC-SA 4.0
\end{verbatim}

\end{document}
