% Options for packages loaded elsewhere
\PassOptionsToPackage{unicode}{hyperref}
\PassOptionsToPackage{hyphens}{url}
\documentclass[
]{article}
\usepackage{xcolor}
\usepackage[margin=1in]{geometry}
\usepackage{amsmath,amssymb}
\setcounter{secnumdepth}{-\maxdimen} % remove section numbering
\usepackage{iftex}
\ifPDFTeX
  \usepackage[T1]{fontenc}
  \usepackage[utf8]{inputenc}
  \usepackage{textcomp} % provide euro and other symbols
\else % if luatex or xetex
  \usepackage{unicode-math} % this also loads fontspec
  \defaultfontfeatures{Scale=MatchLowercase}
  \defaultfontfeatures[\rmfamily]{Ligatures=TeX,Scale=1}
\fi
\usepackage{lmodern}
\ifPDFTeX\else
  % xetex/luatex font selection
\fi
% Use upquote if available, for straight quotes in verbatim environments
\IfFileExists{upquote.sty}{\usepackage{upquote}}{}
\IfFileExists{microtype.sty}{% use microtype if available
  \usepackage[]{microtype}
  \UseMicrotypeSet[protrusion]{basicmath} % disable protrusion for tt fonts
}{}
\makeatletter
\@ifundefined{KOMAClassName}{% if non-KOMA class
  \IfFileExists{parskip.sty}{%
    \usepackage{parskip}
  }{% else
    \setlength{\parindent}{0pt}
    \setlength{\parskip}{6pt plus 2pt minus 1pt}}
}{% if KOMA class
  \KOMAoptions{parskip=half}}
\makeatother
\usepackage{color}
\usepackage{fancyvrb}
\newcommand{\VerbBar}{|}
\newcommand{\VERB}{\Verb[commandchars=\\\{\}]}
\DefineVerbatimEnvironment{Highlighting}{Verbatim}{commandchars=\\\{\}}
% Add ',fontsize=\small' for more characters per line
\usepackage{framed}
\definecolor{shadecolor}{RGB}{248,248,248}
\newenvironment{Shaded}{\begin{snugshade}}{\end{snugshade}}
\newcommand{\AlertTok}[1]{\textcolor[rgb]{0.94,0.16,0.16}{#1}}
\newcommand{\AnnotationTok}[1]{\textcolor[rgb]{0.56,0.35,0.01}{\textbf{\textit{#1}}}}
\newcommand{\AttributeTok}[1]{\textcolor[rgb]{0.13,0.29,0.53}{#1}}
\newcommand{\BaseNTok}[1]{\textcolor[rgb]{0.00,0.00,0.81}{#1}}
\newcommand{\BuiltInTok}[1]{#1}
\newcommand{\CharTok}[1]{\textcolor[rgb]{0.31,0.60,0.02}{#1}}
\newcommand{\CommentTok}[1]{\textcolor[rgb]{0.56,0.35,0.01}{\textit{#1}}}
\newcommand{\CommentVarTok}[1]{\textcolor[rgb]{0.56,0.35,0.01}{\textbf{\textit{#1}}}}
\newcommand{\ConstantTok}[1]{\textcolor[rgb]{0.56,0.35,0.01}{#1}}
\newcommand{\ControlFlowTok}[1]{\textcolor[rgb]{0.13,0.29,0.53}{\textbf{#1}}}
\newcommand{\DataTypeTok}[1]{\textcolor[rgb]{0.13,0.29,0.53}{#1}}
\newcommand{\DecValTok}[1]{\textcolor[rgb]{0.00,0.00,0.81}{#1}}
\newcommand{\DocumentationTok}[1]{\textcolor[rgb]{0.56,0.35,0.01}{\textbf{\textit{#1}}}}
\newcommand{\ErrorTok}[1]{\textcolor[rgb]{0.64,0.00,0.00}{\textbf{#1}}}
\newcommand{\ExtensionTok}[1]{#1}
\newcommand{\FloatTok}[1]{\textcolor[rgb]{0.00,0.00,0.81}{#1}}
\newcommand{\FunctionTok}[1]{\textcolor[rgb]{0.13,0.29,0.53}{\textbf{#1}}}
\newcommand{\ImportTok}[1]{#1}
\newcommand{\InformationTok}[1]{\textcolor[rgb]{0.56,0.35,0.01}{\textbf{\textit{#1}}}}
\newcommand{\KeywordTok}[1]{\textcolor[rgb]{0.13,0.29,0.53}{\textbf{#1}}}
\newcommand{\NormalTok}[1]{#1}
\newcommand{\OperatorTok}[1]{\textcolor[rgb]{0.81,0.36,0.00}{\textbf{#1}}}
\newcommand{\OtherTok}[1]{\textcolor[rgb]{0.56,0.35,0.01}{#1}}
\newcommand{\PreprocessorTok}[1]{\textcolor[rgb]{0.56,0.35,0.01}{\textit{#1}}}
\newcommand{\RegionMarkerTok}[1]{#1}
\newcommand{\SpecialCharTok}[1]{\textcolor[rgb]{0.81,0.36,0.00}{\textbf{#1}}}
\newcommand{\SpecialStringTok}[1]{\textcolor[rgb]{0.31,0.60,0.02}{#1}}
\newcommand{\StringTok}[1]{\textcolor[rgb]{0.31,0.60,0.02}{#1}}
\newcommand{\VariableTok}[1]{\textcolor[rgb]{0.00,0.00,0.00}{#1}}
\newcommand{\VerbatimStringTok}[1]{\textcolor[rgb]{0.31,0.60,0.02}{#1}}
\newcommand{\WarningTok}[1]{\textcolor[rgb]{0.56,0.35,0.01}{\textbf{\textit{#1}}}}
\usepackage{longtable,booktabs,array}
\usepackage{calc} % for calculating minipage widths
% Correct order of tables after \paragraph or \subparagraph
\usepackage{etoolbox}
\makeatletter
\patchcmd\longtable{\par}{\if@noskipsec\mbox{}\fi\par}{}{}
\makeatother
% Allow footnotes in longtable head/foot
\IfFileExists{footnotehyper.sty}{\usepackage{footnotehyper}}{\usepackage{footnote}}
\makesavenoteenv{longtable}
\usepackage{graphicx}
\makeatletter
\newsavebox\pandoc@box
\newcommand*\pandocbounded[1]{% scales image to fit in text height/width
  \sbox\pandoc@box{#1}%
  \Gscale@div\@tempa{\textheight}{\dimexpr\ht\pandoc@box+\dp\pandoc@box\relax}%
  \Gscale@div\@tempb{\linewidth}{\wd\pandoc@box}%
  \ifdim\@tempb\p@<\@tempa\p@\let\@tempa\@tempb\fi% select the smaller of both
  \ifdim\@tempa\p@<\p@\scalebox{\@tempa}{\usebox\pandoc@box}%
  \else\usebox{\pandoc@box}%
  \fi%
}
% Set default figure placement to htbp
\def\fps@figure{htbp}
\makeatother
\setlength{\emergencystretch}{3em} % prevent overfull lines
\providecommand{\tightlist}{%
  \setlength{\itemsep}{0pt}\setlength{\parskip}{0pt}}
\usepackage{bookmark}
\IfFileExists{xurl.sty}{\usepackage{xurl}}{} % add URL line breaks if available
\urlstyle{same}
\hypersetup{
  pdftitle={Tutorial 2 - Probability distributions},
  pdfauthor={SFR605},
  hidelinks,
  pdfcreator={LaTeX via pandoc}}

\title{Tutorial 2 - Probability distributions}
\author{SFR605}
\date{}

\begin{document}
\maketitle

\subsection{Objectives}\label{objectives}

\begin{itemize}
\tightlist
\item
  Familiarize yourself with common probability distributions and R's
  functions for calling them
\item
  Understand how different probability distributions are related to one
  another
\item
  Understand how discrete distributions can be approximated by
  continuous distributions
\end{itemize}

\subsection{Git updating:}\label{git-updating}

To get new or updated tutorials: * If you originally pulled directly
from the original repository + In RStudio's ``Git'' tab, click on the
down arrow + OR at the command line \texttt{git\ pull} * If you FORKED
the original repo to your Github account + At the command line
\texttt{git\ pull\ git@github.com:rabramoff/SFR605.git} + OR
\texttt{git\ pull\ https://github.com/rabramoff/SFR605.git}

\subsection{Probability distributions in
R}\label{probability-distributions-in-r}

Because it is a statistical language, there are a large number of
probability distributions in R by default and an even larger number that
can be loaded from packages. The table below gives a listing of the most
common distributions in R, the name of the function within R, and the
parameters of the distribution

\subsection{Common distributions}\label{common-distributions}

\begin{longtable}[]{@{}lll@{}}
\toprule\noalign{}
Distribution & R name & Parameters \\
\midrule\noalign{}
\endhead
\bottomrule\noalign{}
\endlastfoot
beta & beta & shape1, shape2, ncp \\
Binomial & binom & size, prob \\
Cauchy & cauchy & location, scale \\
chi-squared & chisq & df, ncp \\
exponential & exp & rate \\
F & f & df1, df2, ncp \\
gamma & gamma & shape, scale \\
geometric & geom & prob \\
hypergeometric & hyper & m, n, k \\
log-normal & lnorm & meanlog, sdlog \\
logistic & logis & location, scale \\
Negative binomial & nbinom & size, prob \\
normal & norm & mean, sd \\
Poisson & pois & lambda \\
Student's t & t & df, ncp \\
uniform & unif & min, max \\
Weibull & weibull & shape, scale \\
Wilcoxon & wilcox & m, n \\
\end{longtable}

R actually provides four related functions for each probability
distribution. These functions are called by adding a letter at the
beginning of the function name. The variants of each probability
distribution are:

\begin{itemize}
\tightlist
\item
  ``d'' = density: probability density function (PDF)
\item
  ``p'' = cumulative distribution function (CDF)
\item
  ``q'' = quantile: calculates the value associated with a specified
  tail probability, inverse of ``p''
\item
  ``r'' = random: simulates random numbers
\end{itemize}

The first argument to these functions is the same regardless of the
distribution and is the value \emph{x} for ``d'', the quantile \emph{q}
for ``p'', the probability \emph{p} for ``q'', and the sample size
\emph{n} for ``r''

All of this will make more sense once we consider a concrete example.
Let's take a look at the normal probability density function first,
since it's the one you're most familiar with. If you use \textbf{?dnorm}
you'll see that for many of the function arguments there are default
values, specifically mean=0 and sd=1. Therefore if these values are not
specified explicitly in the function call R assumes you want a
\emph{standard Normal} distribution.

\begin{Shaded}
\begin{Highlighting}[]
\NormalTok{x }\OtherTok{=} \FunctionTok{seq}\NormalTok{(}\SpecialCharTok{{-}}\DecValTok{5}\NormalTok{,}\DecValTok{5}\NormalTok{,}\AttributeTok{by=}\FloatTok{0.1}\NormalTok{)}
\FunctionTok{plot}\NormalTok{(x,}\FunctionTok{dnorm}\NormalTok{(x),}\AttributeTok{type=}\StringTok{\textquotesingle{}l\textquotesingle{}}\NormalTok{)       }\DocumentationTok{\#\# that’s a lowercase “L” for “line”}
\FunctionTok{abline}\NormalTok{(}\AttributeTok{v=}\DecValTok{0}\NormalTok{)                           }\DocumentationTok{\#\# add a line to indicate the mean (“v” is for “vertical”)}
\FunctionTok{lines}\NormalTok{(x,}\FunctionTok{dnorm}\NormalTok{(x,}\DecValTok{3}\NormalTok{),}\AttributeTok{col=}\DecValTok{2}\NormalTok{)           }\DocumentationTok{\#\# try changing the mean (“col” sets the color)}
\FunctionTok{abline}\NormalTok{(}\AttributeTok{v=}\DecValTok{3}\NormalTok{,}\AttributeTok{col=}\DecValTok{2}\NormalTok{)}
\FunctionTok{lines}\NormalTok{(x,}\FunctionTok{dnorm}\NormalTok{(x,}\SpecialCharTok{{-}}\DecValTok{2}\NormalTok{,}\DecValTok{1}\NormalTok{),}\AttributeTok{col=}\DecValTok{3}\NormalTok{)    }\DocumentationTok{\#\# try changing the mean and standard dev}
\FunctionTok{abline}\NormalTok{(}\AttributeTok{v=}\SpecialCharTok{{-}}\DecValTok{2}\NormalTok{,}\AttributeTok{col=}\DecValTok{3}\NormalTok{)}
\end{Highlighting}
\end{Shaded}

\pandocbounded{\includegraphics[keepaspectratio]{02_Distributions_files/figure-latex/unnamed-chunk-1-1.pdf}}

The above plot can also be produced using R's \texttt{curve} function

\begin{Shaded}
\begin{Highlighting}[]
\FunctionTok{curve}\NormalTok{(dnorm,}\SpecialCharTok{{-}}\DecValTok{5}\NormalTok{,}\DecValTok{5}\NormalTok{)}
\FunctionTok{abline}\NormalTok{(}\AttributeTok{v=}\DecValTok{0}\NormalTok{)}
\FunctionTok{curve}\NormalTok{(}\FunctionTok{dnorm}\NormalTok{(x,}\DecValTok{2}\NormalTok{),}\SpecialCharTok{{-}}\DecValTok{5}\NormalTok{,}\DecValTok{5}\NormalTok{,}\AttributeTok{add=}\ConstantTok{TRUE}\NormalTok{,}\AttributeTok{col=}\DecValTok{2}\NormalTok{)}
\FunctionTok{abline}\NormalTok{(}\AttributeTok{v=}\DecValTok{2}\NormalTok{,}\AttributeTok{col=}\DecValTok{2}\NormalTok{)}
\FunctionTok{curve}\NormalTok{(}\FunctionTok{dnorm}\NormalTok{(x,}\SpecialCharTok{{-}}\DecValTok{2}\NormalTok{,}\DecValTok{1}\NormalTok{),}\SpecialCharTok{{-}}\DecValTok{5}\NormalTok{,}\DecValTok{5}\NormalTok{,}\AttributeTok{add=}\ConstantTok{TRUE}\NormalTok{,}\AttributeTok{col=}\DecValTok{3}\NormalTok{)    }\DocumentationTok{\#\# try changing the mean and standard dev}
\FunctionTok{abline}\NormalTok{(}\AttributeTok{v=}\SpecialCharTok{{-}}\DecValTok{2}\NormalTok{,}\AttributeTok{col=}\DecValTok{3}\NormalTok{)}
\end{Highlighting}
\end{Shaded}

\pandocbounded{\includegraphics[keepaspectratio]{02_Distributions_files/figure-latex/unnamed-chunk-2-1.pdf}}

You are welcome to use either approach in the following examples.

You will use density functions quite frequently to estimate the
\textbf{likelihood} of data. For example if we collected a data point of
x = 0.5 we can calculate the likelihood that this data point came from
each of these three normal distributions. Implicit in the likelihood is
that we're looking at the probability of the data in a very small
window, \(\delta x\), and that \(\delta x\) never changes so that:

\[Pr(x < X < x + \delta x) = \int_{x}^{x + \delta x} f(x)dx \propto f(x)\]

\begin{Shaded}
\begin{Highlighting}[]
\FunctionTok{dnorm}\NormalTok{(}\FloatTok{0.5}\NormalTok{,}\DecValTok{0}\NormalTok{,}\DecValTok{1}\NormalTok{)}
\end{Highlighting}
\end{Shaded}

\begin{verbatim}
## [1] 0.3520653
\end{verbatim}

\begin{Shaded}
\begin{Highlighting}[]
\FunctionTok{dnorm}\NormalTok{(}\FloatTok{0.5}\NormalTok{,}\DecValTok{2}\NormalTok{,}\DecValTok{1}\NormalTok{)}
\end{Highlighting}
\end{Shaded}

\begin{verbatim}
## [1] 0.1295176
\end{verbatim}

\begin{Shaded}
\begin{Highlighting}[]
\FunctionTok{dnorm}\NormalTok{(}\FloatTok{0.5}\NormalTok{,}\SpecialCharTok{{-}}\DecValTok{1}\NormalTok{,}\DecValTok{2}\NormalTok{)}
\end{Highlighting}
\end{Shaded}

\begin{verbatim}
## [1] 0.1505687
\end{verbatim}

This shows that the first distribution has a higher likelihood than the
other two, which are about the same. We interpret this as saying that
the observed data point was more likely to have been generated by the
first distribution than the other two. This is consistent with where a
vertical line centered around 0.5 would intersect each of the curves.

This plot of the normal distribution and the effects of varying the
parameters in the normal are both probably familiar to you already --
changing the mean changes where the distribution is centered while
changing the standard deviation changes the spread of the distribution.
Next try looking at the CDF of the normal:

\begin{Shaded}
\begin{Highlighting}[]
\FunctionTok{plot}\NormalTok{(x,}\FunctionTok{pnorm}\NormalTok{(x,}\DecValTok{0}\NormalTok{,}\DecValTok{1}\NormalTok{),}\AttributeTok{type=}\StringTok{\textquotesingle{}l\textquotesingle{}}\NormalTok{)}
\FunctionTok{abline}\NormalTok{(}\AttributeTok{v=}\DecValTok{0}\NormalTok{)}
\FunctionTok{lines}\NormalTok{(x,}\FunctionTok{pnorm}\NormalTok{(x,}\DecValTok{2}\NormalTok{,}\DecValTok{1}\NormalTok{),}\AttributeTok{col=}\DecValTok{2}\NormalTok{)}
\FunctionTok{abline}\NormalTok{(}\AttributeTok{v=}\DecValTok{2}\NormalTok{,}\AttributeTok{col=}\DecValTok{2}\NormalTok{)}
\FunctionTok{lines}\NormalTok{(x,}\FunctionTok{pnorm}\NormalTok{(x,}\SpecialCharTok{{-}}\DecValTok{1}\NormalTok{,}\DecValTok{2}\NormalTok{),}\AttributeTok{col=}\DecValTok{3}\NormalTok{)}
\FunctionTok{abline}\NormalTok{(}\AttributeTok{v=}\SpecialCharTok{{-}}\DecValTok{1}\NormalTok{,}\AttributeTok{col=}\DecValTok{3}\NormalTok{)}
\end{Highlighting}
\end{Shaded}

\pandocbounded{\includegraphics[keepaspectratio]{02_Distributions_files/figure-latex/unnamed-chunk-4-1.pdf}}

Using the CDF we can calculate \(Pr(X \leq x)\) for our data point

\begin{Shaded}
\begin{Highlighting}[]
\FunctionTok{pnorm}\NormalTok{(}\FloatTok{0.5}\NormalTok{,}\DecValTok{0}\NormalTok{,}\DecValTok{1}\NormalTok{)}
\end{Highlighting}
\end{Shaded}

\begin{verbatim}
## [1] 0.6914625
\end{verbatim}

\begin{Shaded}
\begin{Highlighting}[]
\FunctionTok{pnorm}\NormalTok{(}\FloatTok{0.5}\NormalTok{,}\DecValTok{2}\NormalTok{,}\DecValTok{1}\NormalTok{)}
\end{Highlighting}
\end{Shaded}

\begin{verbatim}
## [1] 0.0668072
\end{verbatim}

\begin{Shaded}
\begin{Highlighting}[]
\FunctionTok{pnorm}\NormalTok{(}\FloatTok{0.5}\NormalTok{,}\SpecialCharTok{{-}}\DecValTok{1}\NormalTok{,}\DecValTok{2}\NormalTok{)}
\end{Highlighting}
\end{Shaded}

\begin{verbatim}
## [1] 0.7733726
\end{verbatim}

If the value 0.5 here corresponded to some hypothesis, the \textbf{CDF
could be used to calculate a p-value associated with the one-sided
test}. Would any be significant at \(\alpha\)=0.05 significance?

\textbf{no}

Next let's look at the function qnorm. Since the input to this function
is a quantile, the x-values for the plot are restricted to the range
{[}0,1{]}.

\begin{Shaded}
\begin{Highlighting}[]
\NormalTok{p }\OtherTok{=} \FunctionTok{seq}\NormalTok{(}\DecValTok{0}\NormalTok{,}\DecValTok{1}\NormalTok{,}\AttributeTok{by=}\FloatTok{0.01}\NormalTok{)}
\FunctionTok{plot}\NormalTok{(p,}\FunctionTok{qnorm}\NormalTok{(p,}\DecValTok{0}\NormalTok{,}\DecValTok{1}\NormalTok{),}\AttributeTok{type=}\StringTok{\textquotesingle{}l\textquotesingle{}}\NormalTok{,}\AttributeTok{ylim=}\FunctionTok{range}\NormalTok{(x))    }\CommentTok{\# ylim sets the y{-}axis range}
\CommentTok{\# range returns the min/max as a 2{-}element vector}
\FunctionTok{abline}\NormalTok{(}\AttributeTok{h=}\DecValTok{0}\NormalTok{)                     }\CommentTok{\# “h” for “horizontal”}
\FunctionTok{lines}\NormalTok{(p,}\FunctionTok{qnorm}\NormalTok{(p,}\DecValTok{2}\NormalTok{,}\DecValTok{1}\NormalTok{),}\AttributeTok{col=}\DecValTok{2}\NormalTok{)}
\FunctionTok{abline}\NormalTok{(}\AttributeTok{h=}\DecValTok{2}\NormalTok{,}\AttributeTok{col=}\DecValTok{2}\NormalTok{)}
\FunctionTok{lines}\NormalTok{(p,}\FunctionTok{qnorm}\NormalTok{(p,}\SpecialCharTok{{-}}\DecValTok{1}\NormalTok{,}\DecValTok{2}\NormalTok{),}\AttributeTok{col=}\DecValTok{3}\NormalTok{)}
\FunctionTok{abline}\NormalTok{(}\AttributeTok{h=}\SpecialCharTok{{-}}\DecValTok{1}\NormalTok{,}\AttributeTok{col=}\DecValTok{3}\NormalTok{)}
\end{Highlighting}
\end{Shaded}

\pandocbounded{\includegraphics[keepaspectratio]{02_Distributions_files/figure-latex/unnamed-chunk-6-1.pdf}}

It should be readily apparent that the quantile function is the inverse
of the CDF. This function can be used to find the median of the
distribution (p = 0.5) or to \textbf{estimate confidence intervals} at
any level desired.

\begin{Shaded}
\begin{Highlighting}[]
\FunctionTok{qnorm}\NormalTok{(}\FunctionTok{c}\NormalTok{(}\FloatTok{0.025}\NormalTok{,}\FloatTok{0.975}\NormalTok{),}\DecValTok{0}\NormalTok{,}\DecValTok{1}\NormalTok{)       }\CommentTok{\# what width CI is specified by these values?}
\end{Highlighting}
\end{Shaded}

\begin{verbatim}
## [1] -1.959964  1.959964
\end{verbatim}

\begin{Shaded}
\begin{Highlighting}[]
\FunctionTok{plot}\NormalTok{(p,}\FunctionTok{qnorm}\NormalTok{(p,}\DecValTok{0}\NormalTok{,}\DecValTok{1}\NormalTok{),}\AttributeTok{type=}\StringTok{\textquotesingle{}l\textquotesingle{}}\NormalTok{,}\AttributeTok{ylim=}\FunctionTok{range}\NormalTok{(x))}
\FunctionTok{abline}\NormalTok{(}\AttributeTok{v=}\FunctionTok{c}\NormalTok{(}\FloatTok{0.025}\NormalTok{,}\FloatTok{0.975}\NormalTok{),}\AttributeTok{lty=}\DecValTok{2}\NormalTok{)  }\CommentTok{\# add vertical lines at the CI}
\FunctionTok{abline}\NormalTok{(}\AttributeTok{h=}\FunctionTok{qnorm}\NormalTok{(}\FunctionTok{c}\NormalTok{(}\FloatTok{0.025}\NormalTok{,}\FloatTok{0.975}\NormalTok{)),}\AttributeTok{lty=}\DecValTok{2}\NormalTok{)   }\CommentTok{\#add horizontal lines at the threshold vals}
\end{Highlighting}
\end{Shaded}

\pandocbounded{\includegraphics[keepaspectratio]{02_Distributions_files/figure-latex/unnamed-chunk-7-1.pdf}}

\begin{Shaded}
\begin{Highlighting}[]
\FunctionTok{plot}\NormalTok{(x,}\FunctionTok{dnorm}\NormalTok{(x,}\DecValTok{0}\NormalTok{,}\DecValTok{1}\NormalTok{),}\AttributeTok{type=}\StringTok{\textquotesingle{}l\textquotesingle{}}\NormalTok{)       }\CommentTok{\# plot the corresponding pdf}
\FunctionTok{abline}\NormalTok{(}\AttributeTok{v=}\FunctionTok{qnorm}\NormalTok{(}\FunctionTok{c}\NormalTok{(}\FloatTok{0.025}\NormalTok{,}\FloatTok{0.975}\NormalTok{)),}\AttributeTok{lty=}\DecValTok{2}\NormalTok{)}
\end{Highlighting}
\end{Shaded}

\pandocbounded{\includegraphics[keepaspectratio]{02_Distributions_files/figure-latex/unnamed-chunk-7-2.pdf}}

Finally, let's investigate the rnorm function for generating random
numbers that have a normal distribution. Here we generate histograms
that have a progressively larger sample size and compare that to the
actual density of the standard normal.

\begin{Shaded}
\begin{Highlighting}[]
\NormalTok{n }\OtherTok{=} \FunctionTok{c}\NormalTok{(}\DecValTok{10}\NormalTok{,}\DecValTok{100}\NormalTok{,}\DecValTok{1000}\NormalTok{,}\DecValTok{10000}\NormalTok{)    }\CommentTok{\# sequence of sample sizes}
\ControlFlowTok{for}\NormalTok{(i }\ControlFlowTok{in} \DecValTok{1}\SpecialCharTok{:}\DecValTok{4}\NormalTok{)\{          }\CommentTok{\# loop over these sample sizes}
  \FunctionTok{hist}\NormalTok{(}\FunctionTok{rnorm}\NormalTok{(n[i]),}\AttributeTok{main=}\NormalTok{n[i],}\AttributeTok{probability=}\ConstantTok{TRUE}\NormalTok{,}\AttributeTok{breaks=}\DecValTok{40}\NormalTok{)  }
                \CommentTok{\#here breaks defines number of bins in the histogram}
  \FunctionTok{lines}\NormalTok{(x,}\FunctionTok{dnorm}\NormalTok{(x),}\AttributeTok{col=}\DecValTok{2}\NormalTok{)}
\NormalTok{\}}
\end{Highlighting}
\end{Shaded}

\pandocbounded{\includegraphics[keepaspectratio]{02_Distributions_files/figure-latex/unnamed-chunk-8-1.pdf}}
\pandocbounded{\includegraphics[keepaspectratio]{02_Distributions_files/figure-latex/unnamed-chunk-8-2.pdf}}
\pandocbounded{\includegraphics[keepaspectratio]{02_Distributions_files/figure-latex/unnamed-chunk-8-3.pdf}}
\pandocbounded{\includegraphics[keepaspectratio]{02_Distributions_files/figure-latex/unnamed-chunk-8-4.pdf}}

To make these plots we introduced the use of a \texttt{for} loop. A for
loop iterates over all specified values of some index. Above, we used
\texttt{i} as our index, and specified values 1, 2, 3, and 4 with the
1:4 syntax from last week. \texttt{for} loops always start this way
(including parentheses): \texttt{for(index\ in\ vector)}. The code in
the \{ \} then executes once for every value in the vector, with
\texttt{i} taking on the next value in the list. The \texttt{for} syntax
in R is very flexible and will allow you to loop over a vector of
anything, not just integers\ldots characters, factors, filenames, etc.
Also note that your index variable can be anything, but to avoid
unwanted results make sure it's not the same as a variable you're using
elsewhere in the code (e.g.~indexing by \texttt{x} above would ruin
everything, because \texttt{x} would be overwritten on every iteration).

One other technical note: like any function in R that generates random
output, this example will give different results every time you run it.

This example demonstrates that \textbf{as the number of random draws
from a probability distribution increases, the histogram of those draws
provides a better and better approximation of the density itself}. We
will make use of this fact extensively this semester because -- as odd
as this may sound now -- there are many distributions that are easier to
randomly sample from than solve for analytically. We can also show that
as the sample size increases, the sample mean and standard deviation get
closer to the true population values (mean = 0, sd = 1 in this example)
and the standard error gets smaller:

\begin{Shaded}
\begin{Highlighting}[]
\NormalTok{y }\OtherTok{=} \FunctionTok{rnorm}\NormalTok{(}\DecValTok{10}\NormalTok{)   }\DocumentationTok{\#\# Sample size = 10}
\NormalTok{y}
\end{Highlighting}
\end{Shaded}

\begin{verbatim}
##  [1]  2.7259365 -0.2437165 -0.2157619 -0.2336910 -1.2010644 -1.2751110
##  [7] -1.8120334 -1.0973363  0.9711838 -1.7216961
\end{verbatim}

\begin{Shaded}
\begin{Highlighting}[]
\FunctionTok{mean}\NormalTok{(y)}
\end{Highlighting}
\end{Shaded}

\begin{verbatim}
## [1] -0.410329
\end{verbatim}

\begin{Shaded}
\begin{Highlighting}[]
\FunctionTok{sd}\NormalTok{(y)           }\DocumentationTok{\#\# standard deviation}
\end{Highlighting}
\end{Shaded}

\begin{verbatim}
## [1] 1.389289
\end{verbatim}

\begin{Shaded}
\begin{Highlighting}[]
\FunctionTok{sd}\NormalTok{(y)}\SpecialCharTok{/}\FunctionTok{sqrt}\NormalTok{(}\DecValTok{10}\NormalTok{)      }\DocumentationTok{\#\# standard error}
\end{Highlighting}
\end{Shaded}

\begin{verbatim}
## [1] 0.4393316
\end{verbatim}

\begin{Shaded}
\begin{Highlighting}[]
\NormalTok{y }\OtherTok{=} \FunctionTok{rnorm}\NormalTok{(}\DecValTok{10000}\NormalTok{)    }\DocumentationTok{\#\# Sample size = 10000}
\FunctionTok{mean}\NormalTok{(y)}
\end{Highlighting}
\end{Shaded}

\begin{verbatim}
## [1] 0.001824008
\end{verbatim}

\begin{Shaded}
\begin{Highlighting}[]
\FunctionTok{sd}\NormalTok{(y)           }\DocumentationTok{\#\# standard deviation}
\end{Highlighting}
\end{Shaded}

\begin{verbatim}
## [1] 0.9968279
\end{verbatim}

\begin{Shaded}
\begin{Highlighting}[]
\FunctionTok{sd}\NormalTok{(y)}\SpecialCharTok{/}\FunctionTok{sqrt}\NormalTok{(}\DecValTok{10000}\NormalTok{)   }\DocumentationTok{\#\# standard error}
\end{Highlighting}
\end{Shaded}

\begin{verbatim}
## [1] 0.009968279
\end{verbatim}

We are next going to investigate several other common probability
distributions. There will be a few examples of the effects of varying
parameters for each distribution, but you are strongly recommended to
explore and try other values to see how they affect the shape of each
PDF. There is a good chart at
\url{http://www.johndcook.com/distribution_chart.html} that describes
the relationships among these common distributions, and the Wikipedia
articles for most of them are good for quick reference.

There's a good bit of code this week. We recommend running each clump of
code by itself, and especially that you \textbf{make sure you understand
what it did before proceeding to the next box}.

\subsubsection{INSTRUCTIONS}\label{instructions}

For each of the distributions below, please provide a short answer to
the questions that follow. If you write any additional code in deriving
your answers (you may not need to in this tutorial), include it and any
R outputs or figures it generates. When doing calculations ``by hand,''
include the formulas you used. You do not have to show all your work,
but it may make it easier for me to understand your reasoning.

\section{Part 1: Continuous
distributions}\label{part-1-continuous-distributions}

\emph{Note, if you didn't start with the tutorial above, you'll need to
define x and p variables before running the codes below.}

\subsubsection{uniform}\label{uniform}

The uniform is used when there's an equal probability of an event
occurring over a range of values.

\begin{Shaded}
\begin{Highlighting}[]
\FunctionTok{plot}\NormalTok{(x,}\FunctionTok{dunif}\NormalTok{(x),}\AttributeTok{type=}\StringTok{\textquotesingle{}l\textquotesingle{}}\NormalTok{)}
\FunctionTok{lines}\NormalTok{(x,}\FunctionTok{dunif}\NormalTok{(x,}\DecValTok{0}\NormalTok{,}\DecValTok{4}\NormalTok{),}\AttributeTok{col=}\DecValTok{2}\NormalTok{)}
\FunctionTok{lines}\NormalTok{(x,}\FunctionTok{dunif}\NormalTok{(x,}\SpecialCharTok{{-}}\DecValTok{3}\NormalTok{,}\SpecialCharTok{{-}}\FloatTok{0.5}\NormalTok{),}\AttributeTok{col=}\DecValTok{3}\NormalTok{)}
\end{Highlighting}
\end{Shaded}

\pandocbounded{\includegraphics[keepaspectratio]{02_Distributions_files/figure-latex/unnamed-chunk-10-1.pdf}}

\begin{Shaded}
\begin{Highlighting}[]
\FunctionTok{plot}\NormalTok{(x,}\FunctionTok{punif}\NormalTok{(x),}\AttributeTok{type=}\StringTok{\textquotesingle{}l\textquotesingle{}}\NormalTok{)}
\FunctionTok{lines}\NormalTok{(x,}\FunctionTok{punif}\NormalTok{(x,}\DecValTok{0}\NormalTok{,}\DecValTok{4}\NormalTok{),}\AttributeTok{col=}\DecValTok{2}\NormalTok{)}
\FunctionTok{lines}\NormalTok{(x,}\FunctionTok{punif}\NormalTok{(x,}\SpecialCharTok{{-}}\DecValTok{3}\NormalTok{,}\SpecialCharTok{{-}}\FloatTok{0.5}\NormalTok{),}\AttributeTok{col=}\DecValTok{3}\NormalTok{)}
\end{Highlighting}
\end{Shaded}

\pandocbounded{\includegraphics[keepaspectratio]{02_Distributions_files/figure-latex/unnamed-chunk-10-2.pdf}}

\begin{Shaded}
\begin{Highlighting}[]
\FunctionTok{plot}\NormalTok{(p,}\FunctionTok{qunif}\NormalTok{(p),}\AttributeTok{type=}\StringTok{\textquotesingle{}l\textquotesingle{}}\NormalTok{,}\AttributeTok{ylim=}\FunctionTok{range}\NormalTok{(x))}
\FunctionTok{lines}\NormalTok{(p,}\FunctionTok{qunif}\NormalTok{(p,}\DecValTok{0}\NormalTok{,}\DecValTok{4}\NormalTok{),}\AttributeTok{col=}\DecValTok{2}\NormalTok{)}
\FunctionTok{lines}\NormalTok{(p,}\FunctionTok{qunif}\NormalTok{(p,}\SpecialCharTok{{-}}\DecValTok{3}\NormalTok{,}\SpecialCharTok{{-}}\FloatTok{0.5}\NormalTok{),}\AttributeTok{col=}\DecValTok{3}\NormalTok{)}
\end{Highlighting}
\end{Shaded}

\pandocbounded{\includegraphics[keepaspectratio]{02_Distributions_files/figure-latex/unnamed-chunk-10-3.pdf}}

\begin{Shaded}
\begin{Highlighting}[]
\FunctionTok{hist}\NormalTok{(}\FunctionTok{runif}\NormalTok{(}\DecValTok{500}\NormalTok{,}\SpecialCharTok{{-}}\DecValTok{3}\NormalTok{,}\DecValTok{3}\NormalTok{),}\AttributeTok{breaks=}\DecValTok{30}\NormalTok{,}\AttributeTok{xlim=}\FunctionTok{range}\NormalTok{(x),}\AttributeTok{probability=}\ConstantTok{TRUE}\NormalTok{)}
\FunctionTok{lines}\NormalTok{(x,}\FunctionTok{dunif}\NormalTok{(x,}\SpecialCharTok{{-}}\DecValTok{3}\NormalTok{,}\DecValTok{3}\NormalTok{),}\AttributeTok{col=}\DecValTok{2}\NormalTok{)}
\end{Highlighting}
\end{Shaded}

\pandocbounded{\includegraphics[keepaspectratio]{02_Distributions_files/figure-latex/unnamed-chunk-10-4.pdf}}

\subsubsection{Questions:}\label{questions}

\begin{verbatim}
1. Why does the height of the uniform PDF change as the width changes?

In order for probability to be equal, it must be divided by the range of values. As that range (the width) grows, the probabilty (height) decreases. If an event will occur between 0 and 1, then there is a 100% chance of it happening in that interval. If it will occure between 0 and 4, then there is a 25% chance of it occuring between 0 and 1, a 25% chance of it happening between 1 and 2 and so on, so the probibility never passes 0.25. 

2. What do the second and third arguments to the uniform specify? What are their default values?

They specify the beginning and end of the range of values. The default values are 0, 1
\end{verbatim}

\subsubsection{Beta}\label{beta}

The Beta is an interesting distribution because it is bound on the range
0 \textless= X \textless= 1 and thus is very often used to describe data
that is a proportion. At first glance the mathematical formula for the
Beta looks a lot like the Binomial:

\[Beta(x \mid a,b) \propto x^{a-1} (1-x)^{b-1}\]
\[Binom(x \mid n,p) \propto p^x (1-p)^{n-p}\]

The critical difference between the two is that in the Beta the random
variable X is in the base while in the Binomial it is in the exponent.
Unlike many distributions you may be have used in the past, the two
shape parameters in the Beta do not define the mean and variance, but
these can be calculated as simple functions of \(\alpha\) and \(\beta\).
The Beta does have an interesting property of symmetry though, whereby
Beta(\(\alpha\), \(\beta\)) is the reflection of
Beta(\(\beta\),\(\alpha\)).

\begin{Shaded}
\begin{Highlighting}[]
\FunctionTok{plot}\NormalTok{(p,}\FunctionTok{dbeta}\NormalTok{(p,}\DecValTok{5}\NormalTok{,}\DecValTok{5}\NormalTok{),}\AttributeTok{type=}\StringTok{\textquotesingle{}l\textquotesingle{}}\NormalTok{)}
\FunctionTok{lines}\NormalTok{(p,}\FunctionTok{dbeta}\NormalTok{(p,}\DecValTok{1}\NormalTok{,}\DecValTok{1}\NormalTok{),}\AttributeTok{col=}\DecValTok{2}\NormalTok{)}
\FunctionTok{lines}\NormalTok{(p,}\FunctionTok{dbeta}\NormalTok{(p,}\FloatTok{0.2}\NormalTok{,}\FloatTok{0.2}\NormalTok{),}\AttributeTok{col=}\DecValTok{3}\NormalTok{)}
\end{Highlighting}
\end{Shaded}

\pandocbounded{\includegraphics[keepaspectratio]{02_Distributions_files/figure-latex/unnamed-chunk-11-1.pdf}}

\begin{Shaded}
\begin{Highlighting}[]
\DocumentationTok{\#\# vary beta}
\FunctionTok{plot}\NormalTok{(p,}\FunctionTok{dbeta}\NormalTok{(p,}\DecValTok{6}\NormalTok{,}\DecValTok{6}\NormalTok{),}\AttributeTok{type=}\StringTok{\textquotesingle{}l\textquotesingle{}}\NormalTok{,}\AttributeTok{ylim=}\FunctionTok{c}\NormalTok{(}\DecValTok{0}\NormalTok{,}\DecValTok{5}\NormalTok{))}
\FunctionTok{lines}\NormalTok{(p,}\FunctionTok{dbeta}\NormalTok{(p,}\DecValTok{6}\NormalTok{,}\DecValTok{4}\NormalTok{),}\AttributeTok{col=}\DecValTok{2}\NormalTok{)}
\FunctionTok{lines}\NormalTok{(p,}\FunctionTok{dbeta}\NormalTok{(p,}\DecValTok{6}\NormalTok{,}\DecValTok{2}\NormalTok{),}\AttributeTok{col=}\DecValTok{3}\NormalTok{)}
\FunctionTok{lines}\NormalTok{(p,}\FunctionTok{dbeta}\NormalTok{(p,}\DecValTok{6}\NormalTok{,}\FloatTok{1.25}\NormalTok{),}\AttributeTok{col=}\DecValTok{4}\NormalTok{)}
\FunctionTok{lines}\NormalTok{(p,}\FunctionTok{dbeta}\NormalTok{(p,}\DecValTok{6}\NormalTok{,}\DecValTok{1}\NormalTok{),}\AttributeTok{col=}\DecValTok{5}\NormalTok{)}
\FunctionTok{lines}\NormalTok{(p,}\FunctionTok{dbeta}\NormalTok{(p,}\DecValTok{6}\NormalTok{,}\FloatTok{0.25}\NormalTok{),}\AttributeTok{col=}\DecValTok{6}\NormalTok{)}
\FunctionTok{legend}\NormalTok{(}\StringTok{"topleft"}\NormalTok{,}\AttributeTok{legend=}\FunctionTok{c}\NormalTok{(}\DecValTok{6}\NormalTok{,}\DecValTok{4}\NormalTok{,}\DecValTok{2}\NormalTok{,}\FloatTok{1.25}\NormalTok{,}\DecValTok{1}\NormalTok{,}\FloatTok{0.5}\NormalTok{),}\AttributeTok{lty=}\DecValTok{1}\NormalTok{,}\AttributeTok{col=}\DecValTok{1}\SpecialCharTok{:}\DecValTok{6}\NormalTok{)}
\end{Highlighting}
\end{Shaded}

\pandocbounded{\includegraphics[keepaspectratio]{02_Distributions_files/figure-latex/unnamed-chunk-11-2.pdf}}

\begin{Shaded}
\begin{Highlighting}[]
\FunctionTok{plot}\NormalTok{(p,}\FunctionTok{pbeta}\NormalTok{(p,}\DecValTok{5}\NormalTok{,}\DecValTok{5}\NormalTok{),}\AttributeTok{type=}\StringTok{\textquotesingle{}l\textquotesingle{}}\NormalTok{)}
\FunctionTok{lines}\NormalTok{(p,}\FunctionTok{pbeta}\NormalTok{(p,}\DecValTok{1}\NormalTok{,}\DecValTok{1}\NormalTok{),}\AttributeTok{col=}\DecValTok{2}\NormalTok{)}
\FunctionTok{lines}\NormalTok{(p,}\FunctionTok{pbeta}\NormalTok{(p,}\FloatTok{0.2}\NormalTok{,}\FloatTok{0.2}\NormalTok{),}\AttributeTok{col=}\DecValTok{3}\NormalTok{)}
\end{Highlighting}
\end{Shaded}

\pandocbounded{\includegraphics[keepaspectratio]{02_Distributions_files/figure-latex/unnamed-chunk-11-3.pdf}}

\begin{Shaded}
\begin{Highlighting}[]
\FunctionTok{hist}\NormalTok{(}\FunctionTok{rbeta}\NormalTok{(}\DecValTok{500}\NormalTok{,}\DecValTok{3}\NormalTok{,}\DecValTok{3}\NormalTok{),}\AttributeTok{breaks=}\DecValTok{30}\NormalTok{,}\AttributeTok{xlim=}\FunctionTok{range}\NormalTok{(p),}\AttributeTok{probability=}\ConstantTok{TRUE}\NormalTok{)}
\FunctionTok{lines}\NormalTok{(p,}\FunctionTok{dbeta}\NormalTok{(p,}\DecValTok{3}\NormalTok{,}\DecValTok{3}\NormalTok{),}\AttributeTok{col=}\DecValTok{2}\NormalTok{)}
\FunctionTok{lines}\NormalTok{(p,}\FunctionTok{dbeta}\NormalTok{(p,}\DecValTok{3}\NormalTok{,}\DecValTok{3}\NormalTok{),}\AttributeTok{col=}\DecValTok{2}\NormalTok{)}
\end{Highlighting}
\end{Shaded}

\pandocbounded{\includegraphics[keepaspectratio]{02_Distributions_files/figure-latex/unnamed-chunk-11-4.pdf}}

\subsubsection{Questions}\label{questions-1}

\begin{verbatim}
  3) The Beta has a special case, Beta(1,1) that is equivalent to what other PDF?
  
  Uniform. At 1,1 beta is a uniform random variable. 
  
  4) In the first panel, the mean is the same for each line (0.5).  What are the variances? (Hint: Calculate this analytically. Look up the distribution online or from the Distributions lecture slides.)
  
From Help Window: the variance is a*b / ((a+b)^2(a+b+1)). Calculations done in code block below. The variances are 1: 0.02272727, 2: 0.08333333, and 3: 0.1785714

  5) In the second panel, what are the means and medians of each of the 6 curves?  (Hint: you'll need to calculate the mean analytically and use one of the variants of R's beta function to find the median.)
  
From Help Window: the mean is a / (a + b). Calculations in code block below. Means: 0.5, 0.6, 0.75, 0.8275862, 0.8571429, 0.96. The quantile 0.5 gives x where 50% of the distibution lies below it or the median. qbeta coded for below giving the medians: 0.5, 0.6069152, 0.77151, 0.8580419, 0.8908987, 0.9922756. 
  
\end{verbatim}

\paragraph{Question 4}\label{question-4}

\begin{Shaded}
\begin{Highlighting}[]
\NormalTok{a }\OtherTok{\textless{}{-}} \DecValTok{5}
\NormalTok{b }\OtherTok{\textless{}{-}} \DecValTok{5}
\NormalTok{a}\SpecialCharTok{*}\NormalTok{b }\SpecialCharTok{/}\NormalTok{ ((a}\SpecialCharTok{+}\NormalTok{b)}\SpecialCharTok{\^{}}\DecValTok{2} \SpecialCharTok{*}\NormalTok{ (a}\SpecialCharTok{+}\NormalTok{b}\SpecialCharTok{+}\DecValTok{1}\NormalTok{))}
\end{Highlighting}
\end{Shaded}

\begin{verbatim}
## [1] 0.02272727
\end{verbatim}

\begin{Shaded}
\begin{Highlighting}[]
\NormalTok{a }\OtherTok{\textless{}{-}} \DecValTok{1}
\NormalTok{b }\OtherTok{\textless{}{-}} \DecValTok{1}
\NormalTok{a}\SpecialCharTok{*}\NormalTok{b }\SpecialCharTok{/}\NormalTok{ ((a}\SpecialCharTok{+}\NormalTok{b)}\SpecialCharTok{\^{}}\DecValTok{2} \SpecialCharTok{*}\NormalTok{ (a}\SpecialCharTok{+}\NormalTok{b}\SpecialCharTok{+}\DecValTok{1}\NormalTok{))}
\end{Highlighting}
\end{Shaded}

\begin{verbatim}
## [1] 0.08333333
\end{verbatim}

\begin{Shaded}
\begin{Highlighting}[]
\NormalTok{a }\OtherTok{\textless{}{-}} \FloatTok{0.2}
\NormalTok{b }\OtherTok{\textless{}{-}} \FloatTok{0.2}
\NormalTok{a}\SpecialCharTok{*}\NormalTok{b }\SpecialCharTok{/}\NormalTok{ ((a}\SpecialCharTok{+}\NormalTok{b)}\SpecialCharTok{\^{}}\DecValTok{2} \SpecialCharTok{*}\NormalTok{ (a}\SpecialCharTok{+}\NormalTok{b}\SpecialCharTok{+}\DecValTok{1}\NormalTok{))}
\end{Highlighting}
\end{Shaded}

\begin{verbatim}
## [1] 0.1785714
\end{verbatim}

\paragraph{Question 5}\label{question-5}

\subparagraph{Means}\label{means}

\begin{Shaded}
\begin{Highlighting}[]
\NormalTok{a }\OtherTok{\textless{}{-}} \DecValTok{6}
\NormalTok{b }\OtherTok{\textless{}{-}} \DecValTok{6}
\NormalTok{a }\SpecialCharTok{/}\NormalTok{ (a}\SpecialCharTok{+}\NormalTok{b)}
\end{Highlighting}
\end{Shaded}

\begin{verbatim}
## [1] 0.5
\end{verbatim}

\begin{Shaded}
\begin{Highlighting}[]
\NormalTok{b }\OtherTok{\textless{}{-}} \DecValTok{4}
\NormalTok{a }\SpecialCharTok{/}\NormalTok{ (a}\SpecialCharTok{+}\NormalTok{b)}
\end{Highlighting}
\end{Shaded}

\begin{verbatim}
## [1] 0.6
\end{verbatim}

\begin{Shaded}
\begin{Highlighting}[]
\NormalTok{b }\OtherTok{\textless{}{-}} \DecValTok{2}
\NormalTok{a }\SpecialCharTok{/}\NormalTok{ (a}\SpecialCharTok{+}\NormalTok{b)}
\end{Highlighting}
\end{Shaded}

\begin{verbatim}
## [1] 0.75
\end{verbatim}

\begin{Shaded}
\begin{Highlighting}[]
\NormalTok{b }\OtherTok{\textless{}{-}} \FloatTok{1.25}
\NormalTok{a }\SpecialCharTok{/}\NormalTok{ (a}\SpecialCharTok{+}\NormalTok{b)}
\end{Highlighting}
\end{Shaded}

\begin{verbatim}
## [1] 0.8275862
\end{verbatim}

\begin{Shaded}
\begin{Highlighting}[]
\NormalTok{b }\OtherTok{\textless{}{-}} \DecValTok{1}
\NormalTok{a }\SpecialCharTok{/}\NormalTok{ (a}\SpecialCharTok{+}\NormalTok{b)}
\end{Highlighting}
\end{Shaded}

\begin{verbatim}
## [1] 0.8571429
\end{verbatim}

\begin{Shaded}
\begin{Highlighting}[]
\NormalTok{b }\OtherTok{\textless{}{-}} \FloatTok{0.25}
\NormalTok{a }\SpecialCharTok{/}\NormalTok{ (a}\SpecialCharTok{+}\NormalTok{b)}
\end{Highlighting}
\end{Shaded}

\begin{verbatim}
## [1] 0.96
\end{verbatim}

\subparagraph{Medians}\label{medians}

\begin{Shaded}
\begin{Highlighting}[]
\FunctionTok{qbeta}\NormalTok{(}\FloatTok{0.5}\NormalTok{, }\DecValTok{6}\NormalTok{, }\DecValTok{6}\NormalTok{)}
\end{Highlighting}
\end{Shaded}

\begin{verbatim}
## [1] 0.5
\end{verbatim}

\begin{Shaded}
\begin{Highlighting}[]
\FunctionTok{qbeta}\NormalTok{(}\FloatTok{0.5}\NormalTok{, }\DecValTok{6}\NormalTok{, }\DecValTok{4}\NormalTok{)}
\end{Highlighting}
\end{Shaded}

\begin{verbatim}
## [1] 0.6069152
\end{verbatim}

\begin{Shaded}
\begin{Highlighting}[]
\FunctionTok{qbeta}\NormalTok{(}\FloatTok{0.5}\NormalTok{, }\DecValTok{6}\NormalTok{, }\DecValTok{2}\NormalTok{)}
\end{Highlighting}
\end{Shaded}

\begin{verbatim}
## [1] 0.77151
\end{verbatim}

\begin{Shaded}
\begin{Highlighting}[]
\FunctionTok{qbeta}\NormalTok{(}\FloatTok{0.5}\NormalTok{, }\DecValTok{6}\NormalTok{, }\FloatTok{1.25}\NormalTok{)}
\end{Highlighting}
\end{Shaded}

\begin{verbatim}
## [1] 0.8580419
\end{verbatim}

\begin{Shaded}
\begin{Highlighting}[]
\FunctionTok{qbeta}\NormalTok{(}\FloatTok{0.5}\NormalTok{, }\DecValTok{6}\NormalTok{, }\DecValTok{1}\NormalTok{)}
\end{Highlighting}
\end{Shaded}

\begin{verbatim}
## [1] 0.8908987
\end{verbatim}

\begin{Shaded}
\begin{Highlighting}[]
\FunctionTok{qbeta}\NormalTok{(}\FloatTok{0.5}\NormalTok{, }\DecValTok{6}\NormalTok{, }\FloatTok{0.25}\NormalTok{)}
\end{Highlighting}
\end{Shaded}

\begin{verbatim}
## [1] 0.9922756
\end{verbatim}

\subsubsection{Log Normal}\label{log-normal}

The lognormal is a log transform of the normal distribution. It is
defined on the range X \textgreater{} 0 so is commonly used for data
that cannot be negative by definition. The distribution is also
positively skewed so is often used for skewed data. One thing that often
goes unappreciated with the log-normal is that the mean, E{[}X{]},
depends on the variance:

\[E[X] = e^{\mu + {{1}\over{2}}\sigma^2}\]

This applies not just when you explicitly use the lognormal, but also
\textbf{whenever you log-transform data} and then calculate a mean or
standard deviation -- a fact that is vastly under-appreciated in the
biological and environmental sciences and frequently missed in the
published literature. In fact, ANY data transformation applied to make
data ``more normal'' will change the mean, with the functional form of
the bias depending on the transformation used. You can not simply
back-transform the data without correcting for this bias. This phenomena
is another illustration of Jensen's Inequality.

\begin{Shaded}
\begin{Highlighting}[]
\DocumentationTok{\#\# changing the mean}
\NormalTok{x }\OtherTok{\textless{}{-}} \DecValTok{10}\SpecialCharTok{\^{}}\FunctionTok{seq}\NormalTok{(}\SpecialCharTok{{-}}\DecValTok{2}\NormalTok{,}\DecValTok{2}\NormalTok{,}\AttributeTok{by=}\FloatTok{0.01}\NormalTok{)}
\FunctionTok{plot}\NormalTok{(x,}\FunctionTok{dlnorm}\NormalTok{(x,}\DecValTok{0}\NormalTok{),}\AttributeTok{type=}\StringTok{\textquotesingle{}l\textquotesingle{}}\NormalTok{,}\AttributeTok{xlim=}\FunctionTok{c}\NormalTok{(}\DecValTok{0}\NormalTok{,}\DecValTok{15}\NormalTok{),}\AttributeTok{main=}\StringTok{"Changing the Mean"}\NormalTok{)}
\FunctionTok{lines}\NormalTok{(x,}\FunctionTok{dlnorm}\NormalTok{(x,}\DecValTok{1}\NormalTok{),}\AttributeTok{col=}\DecValTok{2}\NormalTok{)}
\FunctionTok{lines}\NormalTok{(x,}\FunctionTok{dlnorm}\NormalTok{(x,}\DecValTok{2}\NormalTok{),}\AttributeTok{col=}\DecValTok{3}\NormalTok{)}
\FunctionTok{legend}\NormalTok{(}\StringTok{"topright"}\NormalTok{,}\AttributeTok{legend=}\DecValTok{0}\SpecialCharTok{:}\DecValTok{2}\NormalTok{,}\AttributeTok{lty=}\DecValTok{1}\NormalTok{,}\AttributeTok{col=}\DecValTok{1}\SpecialCharTok{:}\DecValTok{3}\NormalTok{)}
\end{Highlighting}
\end{Shaded}

\pandocbounded{\includegraphics[keepaspectratio]{02_Distributions_files/figure-latex/unnamed-chunk-15-1.pdf}}

\begin{Shaded}
\begin{Highlighting}[]
\DocumentationTok{\#\# on a log scale}
\FunctionTok{plot}\NormalTok{(x,}\FunctionTok{dlnorm}\NormalTok{(x,}\DecValTok{0}\NormalTok{),}\AttributeTok{type=}\StringTok{\textquotesingle{}l\textquotesingle{}}\NormalTok{,}\AttributeTok{log=}\StringTok{\textquotesingle{}x\textquotesingle{}}\NormalTok{,}\AttributeTok{main=}\StringTok{"Log Scale"}\NormalTok{)}
\FunctionTok{lines}\NormalTok{(x,}\FunctionTok{dlnorm}\NormalTok{(x,}\DecValTok{1}\NormalTok{),}\AttributeTok{col=}\DecValTok{2}\NormalTok{)}
\FunctionTok{lines}\NormalTok{(x,}\FunctionTok{dlnorm}\NormalTok{(x,}\DecValTok{2}\NormalTok{),}\AttributeTok{col=}\DecValTok{3}\NormalTok{)}
\FunctionTok{abline}\NormalTok{(}\AttributeTok{v=}\FunctionTok{exp}\NormalTok{(}\DecValTok{0}\NormalTok{),}\AttributeTok{col=}\DecValTok{1}\NormalTok{)}
\FunctionTok{abline}\NormalTok{(}\AttributeTok{v=}\FunctionTok{exp}\NormalTok{(}\DecValTok{1}\NormalTok{),}\AttributeTok{col=}\DecValTok{2}\NormalTok{)}
\FunctionTok{abline}\NormalTok{(}\AttributeTok{v=}\FunctionTok{exp}\NormalTok{(}\DecValTok{2}\NormalTok{),}\AttributeTok{col=}\DecValTok{3}\NormalTok{)}
\FunctionTok{legend}\NormalTok{(}\StringTok{"topright"}\NormalTok{,}\AttributeTok{legend=}\DecValTok{0}\SpecialCharTok{:}\DecValTok{2}\NormalTok{,}\AttributeTok{lty=}\DecValTok{1}\NormalTok{,}\AttributeTok{col=}\DecValTok{1}\SpecialCharTok{:}\DecValTok{3}\NormalTok{)}
\end{Highlighting}
\end{Shaded}

\pandocbounded{\includegraphics[keepaspectratio]{02_Distributions_files/figure-latex/unnamed-chunk-15-2.pdf}}

\begin{Shaded}
\begin{Highlighting}[]
\DocumentationTok{\#\# changing the variance}
\FunctionTok{plot}\NormalTok{(x,}\FunctionTok{dlnorm}\NormalTok{(x,}\DecValTok{2}\NormalTok{,.}\DecValTok{125}\NormalTok{),}\AttributeTok{type=}\StringTok{\textquotesingle{}l\textquotesingle{}}\NormalTok{,}\AttributeTok{xlim=}\FunctionTok{c}\NormalTok{(}\DecValTok{0}\NormalTok{,}\DecValTok{20}\NormalTok{),}\AttributeTok{ylim=}\FunctionTok{c}\NormalTok{(}\DecValTok{0}\NormalTok{,}\FloatTok{0.6}\NormalTok{),}\AttributeTok{main=}\StringTok{"Changing the Variance"}\NormalTok{)}
\FunctionTok{lines}\NormalTok{(x,}\FunctionTok{dlnorm}\NormalTok{(x,}\DecValTok{2}\NormalTok{,}\FloatTok{0.25}\NormalTok{),}\AttributeTok{col=}\DecValTok{2}\NormalTok{)}
\FunctionTok{lines}\NormalTok{(x,}\FunctionTok{dlnorm}\NormalTok{(x,}\DecValTok{2}\NormalTok{,}\FloatTok{0.5}\NormalTok{),}\AttributeTok{col=}\DecValTok{3}\NormalTok{)}
\FunctionTok{lines}\NormalTok{(x,}\FunctionTok{dlnorm}\NormalTok{(x,}\DecValTok{2}\NormalTok{,}\DecValTok{1}\NormalTok{),}\AttributeTok{col=}\DecValTok{4}\NormalTok{)}
\FunctionTok{lines}\NormalTok{(x,}\FunctionTok{dlnorm}\NormalTok{(x,}\DecValTok{2}\NormalTok{,}\DecValTok{2}\NormalTok{),}\AttributeTok{col=}\DecValTok{5}\NormalTok{)}
\FunctionTok{lines}\NormalTok{(x,}\FunctionTok{dlnorm}\NormalTok{(x,}\DecValTok{2}\NormalTok{,}\DecValTok{4}\NormalTok{),}\AttributeTok{col=}\DecValTok{6}\NormalTok{)}
\FunctionTok{abline}\NormalTok{(}\AttributeTok{v=}\FunctionTok{exp}\NormalTok{(}\DecValTok{2}\NormalTok{),}\AttributeTok{col=}\DecValTok{1}\NormalTok{)}
\FunctionTok{legend}\NormalTok{(}\StringTok{"topright"}\NormalTok{,}\AttributeTok{legend=}\FunctionTok{c}\NormalTok{(}\FloatTok{0.125}\NormalTok{,}\FloatTok{0.25}\NormalTok{,}\FloatTok{0.5}\NormalTok{,}\DecValTok{1}\NormalTok{,}\DecValTok{2}\NormalTok{,}\DecValTok{4}\NormalTok{),}\AttributeTok{lty=}\DecValTok{1}\NormalTok{,}\AttributeTok{col=}\DecValTok{1}\SpecialCharTok{:}\DecValTok{6}\NormalTok{)}
\end{Highlighting}
\end{Shaded}

\pandocbounded{\includegraphics[keepaspectratio]{02_Distributions_files/figure-latex/unnamed-chunk-15-3.pdf}}

\begin{Shaded}
\begin{Highlighting}[]
\DocumentationTok{\#\# random sample}
\FunctionTok{hist}\NormalTok{(}\FunctionTok{rlnorm}\NormalTok{(}\DecValTok{250}\NormalTok{,}\DecValTok{2}\NormalTok{,}\DecValTok{1}\NormalTok{),}\AttributeTok{breaks=}\DecValTok{30}\NormalTok{,}\AttributeTok{probability=}\ConstantTok{TRUE}\NormalTok{)}
\FunctionTok{lines}\NormalTok{(x,}\FunctionTok{dlnorm}\NormalTok{(x,}\DecValTok{2}\NormalTok{,}\DecValTok{1}\NormalTok{),}\AttributeTok{col=}\DecValTok{4}\NormalTok{)}
\end{Highlighting}
\end{Shaded}

\pandocbounded{\includegraphics[keepaspectratio]{02_Distributions_files/figure-latex/unnamed-chunk-15-4.pdf}}

\subsubsection{Questions}\label{questions-2}

\begin{verbatim}
6) What are the arithmetric and geometric means of the three curves in the first panel? (You can look up the equations for each.)

From help window mean is E(X)=exp(μ+1/2σ2). This is used to calculate the arthmetric mean in the code block below. Default sdlog = 1. Arithmetric means: 1.648721, 4.481689, 12.18249. The median of the lognormal distibution is the same as geometric mean which is natural log ^ mean. Calculations in code block below. Geometric means: 1, 2.718282 (e), 7.389056.
\end{verbatim}

\paragraph{Question 6}\label{question-6}

\subparagraph{Arithmetric means}\label{arithmetric-means}

\begin{Shaded}
\begin{Highlighting}[]
\FunctionTok{exp}\NormalTok{(}\DecValTok{0} \SpecialCharTok{+} \DecValTok{1} \SpecialCharTok{/}\NormalTok{ (}\DecValTok{2} \SpecialCharTok{*} \DecValTok{1}\SpecialCharTok{\^{}}\DecValTok{2}\NormalTok{))}
\end{Highlighting}
\end{Shaded}

\begin{verbatim}
## [1] 1.648721
\end{verbatim}

\begin{Shaded}
\begin{Highlighting}[]
\FunctionTok{exp}\NormalTok{(}\DecValTok{1} \SpecialCharTok{+} \DecValTok{1} \SpecialCharTok{/}\NormalTok{ (}\DecValTok{2} \SpecialCharTok{*} \DecValTok{1}\SpecialCharTok{\^{}}\DecValTok{2}\NormalTok{))}
\end{Highlighting}
\end{Shaded}

\begin{verbatim}
## [1] 4.481689
\end{verbatim}

\begin{Shaded}
\begin{Highlighting}[]
\FunctionTok{exp}\NormalTok{(}\DecValTok{2} \SpecialCharTok{+} \DecValTok{1} \SpecialCharTok{/}\NormalTok{ (}\DecValTok{2} \SpecialCharTok{*} \DecValTok{1}\SpecialCharTok{\^{}}\DecValTok{2}\NormalTok{))}
\end{Highlighting}
\end{Shaded}

\begin{verbatim}
## [1] 12.18249
\end{verbatim}

\subparagraph{Geometric means}\label{geometric-means}

\begin{Shaded}
\begin{Highlighting}[]
\FunctionTok{exp}\NormalTok{(}\DecValTok{0}\NormalTok{)}
\end{Highlighting}
\end{Shaded}

\begin{verbatim}
## [1] 1
\end{verbatim}

\begin{Shaded}
\begin{Highlighting}[]
\FunctionTok{exp}\NormalTok{(}\DecValTok{1}\NormalTok{)}
\end{Highlighting}
\end{Shaded}

\begin{verbatim}
## [1] 2.718282
\end{verbatim}

\begin{Shaded}
\begin{Highlighting}[]
\FunctionTok{exp}\NormalTok{(}\DecValTok{2}\NormalTok{)}
\end{Highlighting}
\end{Shaded}

\begin{verbatim}
## [1] 7.389056
\end{verbatim}

\subsection{Exponential \& Laplace}\label{exponential-laplace}

The exponential distribution arises naturally as the time it takes for
an event to occur when the average rate of occurrence, \emph{r}, is
constant. The exponential is a special case of the Gamma (discussed
next) where \(Exp(X \mid r) = Gamma(X \mid 1,r)\). The exponential is
also a special case of the Weibull,
\(Exp(X \mid r) = Weibull(X \mid r,1)\), where the Weibull is a
generalization of the exponential that allows the rate parameter
\emph{r} to increase or decrease with time. The Laplace is basically a
two-sided exponential and arises naturally if one is dealing with
absolute deviation, \(|x-m|\), rather than squared deviation,
\((x-m)^2\), as is done with the normal.

\begin{Shaded}
\begin{Highlighting}[]
\DocumentationTok{\#\# changing the mean}
\NormalTok{x }\OtherTok{\textless{}{-}} \FunctionTok{seq}\NormalTok{(}\DecValTok{0}\NormalTok{,}\DecValTok{10}\NormalTok{,}\AttributeTok{by=}\FloatTok{0.01}\NormalTok{)}
\FunctionTok{plot}\NormalTok{(x,}\FunctionTok{dexp}\NormalTok{(x,}\FloatTok{0.125}\NormalTok{),}\AttributeTok{type=}\StringTok{\textquotesingle{}l\textquotesingle{}}\NormalTok{,}\AttributeTok{ylim=}\FunctionTok{c}\NormalTok{(}\DecValTok{0}\NormalTok{,}\DecValTok{1}\NormalTok{))}
\FunctionTok{lines}\NormalTok{(x,}\FunctionTok{dexp}\NormalTok{(x,}\FloatTok{0.25}\NormalTok{),}\AttributeTok{col=}\DecValTok{2}\NormalTok{)}
\FunctionTok{lines}\NormalTok{(x,}\FunctionTok{dexp}\NormalTok{(x,}\FloatTok{0.5}\NormalTok{),}\AttributeTok{col=}\DecValTok{3}\NormalTok{)}
\FunctionTok{lines}\NormalTok{(x,}\FunctionTok{dexp}\NormalTok{(x,}\DecValTok{1}\NormalTok{),}\AttributeTok{col=}\DecValTok{4}\NormalTok{)}
\FunctionTok{lines}\NormalTok{(x,}\FunctionTok{dexp}\NormalTok{(x,}\DecValTok{2}\NormalTok{),}\AttributeTok{col=}\DecValTok{5}\NormalTok{)}
\FunctionTok{lines}\NormalTok{(x,}\FunctionTok{dexp}\NormalTok{(x,}\DecValTok{4}\NormalTok{),}\AttributeTok{col=}\DecValTok{6}\NormalTok{)}
\FunctionTok{legend}\NormalTok{(}\StringTok{"topright"}\NormalTok{,}\AttributeTok{legend=}\FunctionTok{c}\NormalTok{(}\FloatTok{0.125}\NormalTok{,}\FloatTok{0.25}\NormalTok{,}\FloatTok{0.5}\NormalTok{,}\DecValTok{1}\NormalTok{,}\DecValTok{2}\NormalTok{,}\DecValTok{4}\NormalTok{),}\AttributeTok{lty=}\DecValTok{1}\NormalTok{,}\AttributeTok{col=}\DecValTok{1}\SpecialCharTok{:}\DecValTok{6}\NormalTok{)}
\end{Highlighting}
\end{Shaded}

\pandocbounded{\includegraphics[keepaspectratio]{02_Distributions_files/figure-latex/unnamed-chunk-18-1.pdf}}

\begin{Shaded}
\begin{Highlighting}[]
\DocumentationTok{\#\# random sample}
\FunctionTok{hist}\NormalTok{(}\FunctionTok{rexp}\NormalTok{(}\DecValTok{250}\NormalTok{,}\DecValTok{2}\NormalTok{),}\AttributeTok{breaks=}\DecValTok{30}\NormalTok{,}\AttributeTok{probability=}\ConstantTok{TRUE}\NormalTok{)}
\FunctionTok{lines}\NormalTok{(x,}\FunctionTok{dexp}\NormalTok{(x,}\DecValTok{2}\NormalTok{),}\AttributeTok{col=}\DecValTok{4}\NormalTok{)}
\end{Highlighting}
\end{Shaded}

\pandocbounded{\includegraphics[keepaspectratio]{02_Distributions_files/figure-latex/unnamed-chunk-18-2.pdf}}

\begin{Shaded}
\begin{Highlighting}[]
\DocumentationTok{\#\# laplace vs Gaussian}
\FunctionTok{plot}\NormalTok{(x,}\FunctionTok{dexp}\NormalTok{(}\FunctionTok{abs}\NormalTok{(x}\DecValTok{{-}5}\NormalTok{),}\DecValTok{1}\NormalTok{)}\SpecialCharTok{/}\DecValTok{2}\NormalTok{,}\AttributeTok{type=}\StringTok{\textquotesingle{}l\textquotesingle{}}\NormalTok{)}
\FunctionTok{lines}\NormalTok{(x,}\FunctionTok{dnorm}\NormalTok{(x,}\DecValTok{5}\NormalTok{),}\AttributeTok{col=}\DecValTok{2}\NormalTok{)}
\end{Highlighting}
\end{Shaded}

\pandocbounded{\includegraphics[keepaspectratio]{02_Distributions_files/figure-latex/unnamed-chunk-18-3.pdf}}

\begin{Shaded}
\begin{Highlighting}[]
\FunctionTok{plot}\NormalTok{(x,}\FunctionTok{dexp}\NormalTok{(}\FunctionTok{abs}\NormalTok{(x}\DecValTok{{-}5}\NormalTok{),}\DecValTok{1}\NormalTok{)}\SpecialCharTok{/}\DecValTok{2}\NormalTok{,}\AttributeTok{type=}\StringTok{\textquotesingle{}l\textquotesingle{}}\NormalTok{,}\AttributeTok{log=}\StringTok{\textquotesingle{}y\textquotesingle{}}\NormalTok{)   }\DocumentationTok{\#\# same plot as last but on a log scale}
\FunctionTok{lines}\NormalTok{(x,}\FunctionTok{dnorm}\NormalTok{(x,}\DecValTok{5}\NormalTok{),}\AttributeTok{col=}\DecValTok{2}\NormalTok{)}
\end{Highlighting}
\end{Shaded}

\pandocbounded{\includegraphics[keepaspectratio]{02_Distributions_files/figure-latex/unnamed-chunk-18-4.pdf}}

\subsubsection{Questions:}\label{questions-3}

\begin{verbatim}
7) The last two panels compare a normal and a Laplace distribution with the same mean and variance.  How do the two distributions compare?  In particular, compare the difference in the probabilities of extreme events in each distribution. (This answer can be qualitative/descriptive.)

The laplace has a higher probability in the tails, meaning it has a higher probability of extreme events. 
\end{verbatim}

\subsection{Gamma}\label{gamma}

The gamma and inverse-gamma distribution are flexible distributions
defined for positive real numbers. These are frequently used to model
the distribution of variances or precisions (precision = 1/variance), in
which case the shape and rate parameters are related to the sample size
and sum of squares, respectively. The gamma is frequently used in
Bayesian statistics as a prior distribution, and also in mixture
distributions for inflating the variance of another distribution.

\begin{Shaded}
\begin{Highlighting}[]
\NormalTok{x }\OtherTok{\textless{}{-}} \FunctionTok{seq}\NormalTok{(}\DecValTok{0}\NormalTok{,}\DecValTok{10}\NormalTok{,}\AttributeTok{by=}\FloatTok{0.01}\NormalTok{)}

\DocumentationTok{\#\# change rate}
\FunctionTok{plot}\NormalTok{(x,}\FunctionTok{dgamma}\NormalTok{(x,}\DecValTok{3}\NormalTok{,}\DecValTok{3}\NormalTok{),}\AttributeTok{type=}\StringTok{\textquotesingle{}l\textquotesingle{}}\NormalTok{ ,}\AttributeTok{ylim=}\FunctionTok{c}\NormalTok{(}\DecValTok{0}\NormalTok{,}\FloatTok{1.6}\NormalTok{))}
\FunctionTok{lines}\NormalTok{(x,}\FunctionTok{dgamma}\NormalTok{(x,}\DecValTok{3}\NormalTok{,}\DecValTok{1}\NormalTok{),}\AttributeTok{col=}\DecValTok{2}\NormalTok{)}
\FunctionTok{lines}\NormalTok{(x,}\FunctionTok{dgamma}\NormalTok{(x,}\DecValTok{3}\NormalTok{,}\DecValTok{6}\NormalTok{),}\AttributeTok{col=}\DecValTok{3}\NormalTok{)}
\FunctionTok{lines}\NormalTok{(x,}\FunctionTok{dgamma}\NormalTok{(x,}\DecValTok{3}\NormalTok{,}\DecValTok{1}\SpecialCharTok{/}\DecValTok{3}\NormalTok{),}\AttributeTok{col=}\DecValTok{4}\NormalTok{)}
\FunctionTok{legend}\NormalTok{(}\StringTok{"topright"}\NormalTok{,}\AttributeTok{legend=}\FunctionTok{c}\NormalTok{(}\DecValTok{3}\NormalTok{,}\DecValTok{1}\NormalTok{,}\DecValTok{6}\NormalTok{,}\FloatTok{0.33}\NormalTok{),}\AttributeTok{lty=}\DecValTok{1}\NormalTok{,}\AttributeTok{col=}\DecValTok{1}\SpecialCharTok{:}\DecValTok{4}\NormalTok{)}
\end{Highlighting}
\end{Shaded}

\pandocbounded{\includegraphics[keepaspectratio]{02_Distributions_files/figure-latex/unnamed-chunk-19-1.pdf}}

\begin{Shaded}
\begin{Highlighting}[]
\DocumentationTok{\#\# change shape}
\FunctionTok{plot}\NormalTok{(x,}\FunctionTok{dgamma}\NormalTok{(x,}\DecValTok{3}\NormalTok{,}\DecValTok{3}\NormalTok{),}\AttributeTok{type=}\StringTok{\textquotesingle{}l\textquotesingle{}}\NormalTok{)}
\FunctionTok{lines}\NormalTok{(x,}\FunctionTok{dgamma}\NormalTok{(x,}\DecValTok{1}\NormalTok{,}\DecValTok{3}\NormalTok{),}\AttributeTok{col=}\DecValTok{2}\NormalTok{)}
\FunctionTok{lines}\NormalTok{(x,}\FunctionTok{dgamma}\NormalTok{(x,}\DecValTok{6}\NormalTok{,}\DecValTok{3}\NormalTok{),}\AttributeTok{col=}\DecValTok{3}\NormalTok{)}
\FunctionTok{lines}\NormalTok{(x,}\FunctionTok{dgamma}\NormalTok{(x,}\DecValTok{18}\NormalTok{,}\DecValTok{3}\NormalTok{),}\AttributeTok{col=}\DecValTok{4}\NormalTok{)}
\FunctionTok{legend}\NormalTok{(}\StringTok{"topright"}\NormalTok{,}\AttributeTok{legend=}\FunctionTok{c}\NormalTok{(}\DecValTok{3}\NormalTok{,}\DecValTok{1}\NormalTok{,}\DecValTok{6}\NormalTok{,}\DecValTok{18}\NormalTok{),}\AttributeTok{lty=}\DecValTok{1}\NormalTok{,}\AttributeTok{col=}\DecValTok{1}\SpecialCharTok{:}\DecValTok{4}\NormalTok{)}
\end{Highlighting}
\end{Shaded}

\pandocbounded{\includegraphics[keepaspectratio]{02_Distributions_files/figure-latex/unnamed-chunk-19-2.pdf}}

\begin{Shaded}
\begin{Highlighting}[]
\DocumentationTok{\#\# change variance}
\NormalTok{a }\OtherTok{\textless{}{-}} \FunctionTok{c}\NormalTok{(}\DecValTok{20}\NormalTok{,}\DecValTok{15}\NormalTok{,}\DecValTok{10}\NormalTok{,}\DecValTok{5}\NormalTok{,}\FloatTok{2.5}\NormalTok{,}\FloatTok{1.25}\NormalTok{)}
\NormalTok{r }\OtherTok{\textless{}{-}} \FunctionTok{c}\NormalTok{(}\DecValTok{4}\NormalTok{,}\DecValTok{3}\NormalTok{,}\DecValTok{2}\NormalTok{,}\DecValTok{1}\NormalTok{,}\FloatTok{0.5}\NormalTok{,}\FloatTok{0.25}\NormalTok{)}
\FunctionTok{plot}\NormalTok{(x,}\FunctionTok{dgamma}\NormalTok{(x,}\DecValTok{20}\NormalTok{,}\DecValTok{4}\NormalTok{),}\AttributeTok{type=}\StringTok{\textquotesingle{}l\textquotesingle{}}\NormalTok{)}
\ControlFlowTok{for}\NormalTok{(i }\ControlFlowTok{in} \DecValTok{1}\SpecialCharTok{:}\DecValTok{6}\NormalTok{)\{}
  \FunctionTok{lines}\NormalTok{(x,}\FunctionTok{dgamma}\NormalTok{(x,a[i],r[i]),}\AttributeTok{col=}\NormalTok{i)\}}
\NormalTok{var }\OtherTok{=}\NormalTok{ a}\SpecialCharTok{/}\NormalTok{r}\SpecialCharTok{\^{}}\DecValTok{2}
\NormalTok{mean }\OtherTok{=}\NormalTok{  a}\SpecialCharTok{/}\NormalTok{r}
\FunctionTok{legend}\NormalTok{(}\StringTok{"topright"}\NormalTok{,}\AttributeTok{legend=}\FunctionTok{format}\NormalTok{(var,}\AttributeTok{digits=}\DecValTok{3}\NormalTok{),}\AttributeTok{lty=}\DecValTok{1}\NormalTok{,}\AttributeTok{col=}\DecValTok{1}\SpecialCharTok{:}\DecValTok{6}\NormalTok{)}
\end{Highlighting}
\end{Shaded}

\pandocbounded{\includegraphics[keepaspectratio]{02_Distributions_files/figure-latex/unnamed-chunk-19-3.pdf}}

\begin{Shaded}
\begin{Highlighting}[]
\NormalTok{var}
\end{Highlighting}
\end{Shaded}

\begin{verbatim}
## [1]  1.250000  1.666667  2.500000  5.000000 10.000000 20.000000
\end{verbatim}

\begin{Shaded}
\begin{Highlighting}[]
\NormalTok{mean}
\end{Highlighting}
\end{Shaded}

\begin{verbatim}
## [1] 5 5 5 5 5 5
\end{verbatim}

\begin{Shaded}
\begin{Highlighting}[]
\DocumentationTok{\#\# change mean}
\NormalTok{var }\OtherTok{=} \DecValTok{4}
\NormalTok{mean }\OtherTok{=} \FunctionTok{c}\NormalTok{(}\DecValTok{1}\NormalTok{,}\DecValTok{2}\NormalTok{,}\DecValTok{3}\NormalTok{,}\DecValTok{4}\NormalTok{,}\DecValTok{5}\NormalTok{)}
\NormalTok{rate }\OtherTok{=}\NormalTok{ mean}\SpecialCharTok{/}\NormalTok{var}
\NormalTok{shape }\OtherTok{=}\NormalTok{ mean}\SpecialCharTok{\^{}}\DecValTok{2}\SpecialCharTok{/}\NormalTok{var}
\FunctionTok{plot}\NormalTok{(x,}\FunctionTok{dgamma}\NormalTok{(x,shape[}\DecValTok{2}\NormalTok{],rate[}\DecValTok{2}\NormalTok{]),}\AttributeTok{type=}\StringTok{\textquotesingle{}l\textquotesingle{}}\NormalTok{,}\AttributeTok{col=}\DecValTok{2}\NormalTok{)}
\ControlFlowTok{for}\NormalTok{(i }\ControlFlowTok{in} \DecValTok{1}\SpecialCharTok{:}\DecValTok{5}\NormalTok{)\{}
  \FunctionTok{lines}\NormalTok{(x,}\FunctionTok{dgamma}\NormalTok{(x,shape[i],rate[i]),}\AttributeTok{col=}\NormalTok{i)}
\NormalTok{\}}
\FunctionTok{legend}\NormalTok{(}\StringTok{"topright"}\NormalTok{,}\AttributeTok{legend=}\DecValTok{1}\SpecialCharTok{:}\DecValTok{5}\NormalTok{,}\AttributeTok{lty=}\DecValTok{1}\NormalTok{,}\AttributeTok{col=}\DecValTok{1}\SpecialCharTok{:}\DecValTok{5}\NormalTok{)}
\end{Highlighting}
\end{Shaded}

\pandocbounded{\includegraphics[keepaspectratio]{02_Distributions_files/figure-latex/unnamed-chunk-19-4.pdf}}

\begin{Shaded}
\begin{Highlighting}[]
\NormalTok{rate}
\end{Highlighting}
\end{Shaded}

\begin{verbatim}
## [1] 0.25 0.50 0.75 1.00 1.25
\end{verbatim}

\begin{Shaded}
\begin{Highlighting}[]
\NormalTok{shape}
\end{Highlighting}
\end{Shaded}

\begin{verbatim}
## [1] 0.25 1.00 2.25 4.00 6.25
\end{verbatim}

\subsubsection{Questions:}\label{questions-4}

\begin{verbatim}
8) Looking at the 'change variance' figure, how does the variance change as a and r increase? Qualitatively, how does this affect the mode and skew? Quantitatively, how does this affect the median (relative to the mean)?

R^2 is the denominator of the variance formula, so incresing it decreases the variance. a is the numeriator so increasing it increses the variances. As the variance increases, the mode shifts towards 0. The right tail or skew increases. The median is calculated compared to the mean in the code block below. As the variance increases, the median becomes smaller compared to the mean. As r decreases. The lower a and r together lead to a larger difference between the median and the mean. 
\end{verbatim}

\paragraph{Question 8}\label{question-8}

\begin{Shaded}
\begin{Highlighting}[]
\CommentTok{\# vanriance}
\NormalTok{a}\SpecialCharTok{/}\NormalTok{r}\SpecialCharTok{\^{}}\DecValTok{2}
\end{Highlighting}
\end{Shaded}

\begin{verbatim}
## [1]  1.250000  1.666667  2.500000  5.000000 10.000000 20.000000
\end{verbatim}

\begin{Shaded}
\begin{Highlighting}[]
\CommentTok{\#Mean}
\NormalTok{a}\SpecialCharTok{/}\NormalTok{r}
\end{Highlighting}
\end{Shaded}

\begin{verbatim}
## [1] 5 5 5 5 5 5
\end{verbatim}

\begin{Shaded}
\begin{Highlighting}[]
\CommentTok{\#Median}
\FunctionTok{qgamma}\NormalTok{(.}\DecValTok{5}\NormalTok{,a,r)}
\end{Highlighting}
\end{Shaded}

\begin{verbatim}
## [1] 4.916918 4.889339 4.834357 4.670909 4.351460 3.747696
\end{verbatim}

\section{Part 2: Discrete
distributions}\label{part-2-discrete-distributions}

\subsection{Binomial}\label{binomial}

The binomial arises naturally from counts of the number of successes
given a probability of success, p, and a sample size, n.~You are
probably already familiar with the binomial in the context of coin toss
examples.

\begin{Shaded}
\begin{Highlighting}[]
\NormalTok{x }\OtherTok{\textless{}{-}} \DecValTok{0}\SpecialCharTok{:}\DecValTok{11}

\DocumentationTok{\#\# vary size of sample (number of draws)}
\NormalTok{size }\OtherTok{=} \FunctionTok{c}\NormalTok{(}\DecValTok{1}\NormalTok{,}\DecValTok{5}\NormalTok{,}\DecValTok{10}\NormalTok{)}
\FunctionTok{plot}\NormalTok{(x,}\FunctionTok{dbinom}\NormalTok{(x,size[}\DecValTok{1}\NormalTok{],}\FloatTok{0.5}\NormalTok{),}\AttributeTok{type=}\StringTok{\textquotesingle{}s\textquotesingle{}}\NormalTok{)}
\ControlFlowTok{for}\NormalTok{(i }\ControlFlowTok{in} \DecValTok{2}\SpecialCharTok{:}\DecValTok{3}\NormalTok{)\{}
  \FunctionTok{lines}\NormalTok{(x,}\FunctionTok{dbinom}\NormalTok{(x,size[i],}\FloatTok{0.5}\NormalTok{),}\AttributeTok{type=}\StringTok{\textquotesingle{}s\textquotesingle{}}\NormalTok{,}\AttributeTok{col=}\NormalTok{i)}
\NormalTok{\}}
\FunctionTok{legend}\NormalTok{(}\StringTok{"topright"}\NormalTok{,}\AttributeTok{legend=}\NormalTok{size,}\AttributeTok{lty=}\DecValTok{1}\NormalTok{,}\AttributeTok{col=}\DecValTok{1}\SpecialCharTok{:}\DecValTok{3}\NormalTok{)}
\end{Highlighting}
\end{Shaded}

\pandocbounded{\includegraphics[keepaspectratio]{02_Distributions_files/figure-latex/unnamed-chunk-21-1.pdf}}

\begin{Shaded}
\begin{Highlighting}[]
\DocumentationTok{\#\# vary probability}
\NormalTok{n }\OtherTok{=} \DecValTok{10}
\NormalTok{p }\OtherTok{=} \FunctionTok{c}\NormalTok{(}\FloatTok{0.1}\NormalTok{,}\FloatTok{0.5}\NormalTok{,}\FloatTok{0.8}\NormalTok{)}
\FunctionTok{plot}\NormalTok{(x,}\FunctionTok{dbinom}\NormalTok{(x,n,p[}\DecValTok{1}\NormalTok{]),}\AttributeTok{type=}\StringTok{\textquotesingle{}s\textquotesingle{}}\NormalTok{)}
\ControlFlowTok{for}\NormalTok{(i }\ControlFlowTok{in} \DecValTok{2}\SpecialCharTok{:}\DecValTok{3}\NormalTok{)\{}
  \FunctionTok{lines}\NormalTok{(x,}\FunctionTok{dbinom}\NormalTok{(x,n,p[i]),}\AttributeTok{col=}\NormalTok{i,}\AttributeTok{type=}\StringTok{\textquotesingle{}s\textquotesingle{}}\NormalTok{)}
\NormalTok{\}}
\FunctionTok{abline}\NormalTok{(}\AttributeTok{v =}\NormalTok{ n}\SpecialCharTok{*}\NormalTok{p,}\AttributeTok{col=}\DecValTok{1}\SpecialCharTok{:}\DecValTok{3}\NormalTok{,}\AttributeTok{lty=}\DecValTok{2}\NormalTok{)}
\FunctionTok{legend}\NormalTok{(}\StringTok{"topright"}\NormalTok{,}\AttributeTok{legend=}\NormalTok{p,}\AttributeTok{lty=}\DecValTok{1}\NormalTok{,}\AttributeTok{col=}\DecValTok{1}\SpecialCharTok{:}\DecValTok{3}\NormalTok{)}
\end{Highlighting}
\end{Shaded}

\pandocbounded{\includegraphics[keepaspectratio]{02_Distributions_files/figure-latex/unnamed-chunk-21-2.pdf}}

\begin{Shaded}
\begin{Highlighting}[]
\DocumentationTok{\#\# CDF}
\FunctionTok{plot}\NormalTok{(x,}\FunctionTok{pbinom}\NormalTok{(x,n,p[}\DecValTok{1}\NormalTok{]),}\AttributeTok{type=}\StringTok{\textquotesingle{}s\textquotesingle{}}\NormalTok{,}\AttributeTok{ylim=}\FunctionTok{c}\NormalTok{(}\DecValTok{0}\NormalTok{,}\DecValTok{1}\NormalTok{))}
\ControlFlowTok{for}\NormalTok{(i }\ControlFlowTok{in} \DecValTok{2}\SpecialCharTok{:}\DecValTok{3}\NormalTok{)\{}
  \FunctionTok{lines}\NormalTok{(x,}\FunctionTok{pbinom}\NormalTok{(x,n,p[i]),}\AttributeTok{col=}\NormalTok{i,}\AttributeTok{type=}\StringTok{\textquotesingle{}s\textquotesingle{}}\NormalTok{)}
\NormalTok{\}}
\FunctionTok{legend}\NormalTok{(}\StringTok{"bottomright"}\NormalTok{,}\AttributeTok{legend=}\NormalTok{p,}\AttributeTok{lty=}\DecValTok{1}\NormalTok{,}\AttributeTok{col=}\DecValTok{1}\SpecialCharTok{:}\DecValTok{3}\NormalTok{)}
\end{Highlighting}
\end{Shaded}

\pandocbounded{\includegraphics[keepaspectratio]{02_Distributions_files/figure-latex/unnamed-chunk-21-3.pdf}}

\begin{Shaded}
\begin{Highlighting}[]
\DocumentationTok{\#\# Random samples}
\FunctionTok{hist}\NormalTok{(}\FunctionTok{rbinom}\NormalTok{(}\DecValTok{100}\NormalTok{,}\DecValTok{10}\NormalTok{,}\FloatTok{0.5}\NormalTok{)}\SpecialCharTok{+}\FloatTok{0.0001}\NormalTok{,   }\DocumentationTok{\#\# small amount added because}
     \AttributeTok{probability=}\ConstantTok{TRUE}\NormalTok{,       }\DocumentationTok{\#\# of way R calculates breaks }
     \AttributeTok{breaks=}\DecValTok{0}\SpecialCharTok{:}\DecValTok{11}\NormalTok{,}\AttributeTok{main=}\StringTok{"Random Binomial"}\NormalTok{) }
\FunctionTok{lines}\NormalTok{(x,}\FunctionTok{dbinom}\NormalTok{(x,}\DecValTok{10}\NormalTok{,}\FloatTok{0.5}\NormalTok{),}\AttributeTok{type=}\StringTok{\textquotesingle{}s\textquotesingle{}}\NormalTok{,}\AttributeTok{col=}\DecValTok{2}\NormalTok{)    }\CommentTok{\#Analytical solution}
\FunctionTok{legend}\NormalTok{(}\StringTok{"topright"}\NormalTok{,}\AttributeTok{legend=}\FunctionTok{c}\NormalTok{(}\StringTok{"sample"}\NormalTok{,}\StringTok{"pmf"}\NormalTok{),}\AttributeTok{lty=}\DecValTok{1}\NormalTok{,}\AttributeTok{col=}\DecValTok{1}\SpecialCharTok{:}\DecValTok{2}\NormalTok{)}
\end{Highlighting}
\end{Shaded}

\pandocbounded{\includegraphics[keepaspectratio]{02_Distributions_files/figure-latex/unnamed-chunk-21-4.pdf}}

\subsubsection{Questions:}\label{questions-5}

\begin{verbatim}
9)  Consider a binomial distribution that has a constant mean, np.  What are the differences in the shape of this distribution if it has a high n and low p vs. a low n and high p?

Increasing n smooths the curve, while increasing p brings the peek closer to n. A p < .5 has a right skew and p > .5 has a left skew.  If it had a high n and low p it would have a smooth curve and right skew while a low n and high p would have the opposite: jagged curve and left skew. 
\end{verbatim}

\subsection{Poisson}\label{poisson}

The Poisson is also very common for count data and arises as the number
of events that occur in a fixed amount of time (e.g.~number of bird
sightings per hour), or the number of items found in a fixed amount of
space (e.g.~the number of trees in a plot). Unlike the Binomial
distribution the Poisson doesn't have a fixed upper bound for the number
of events that can occur.

\begin{Shaded}
\begin{Highlighting}[]
\NormalTok{x }\OtherTok{\textless{}{-}} \DecValTok{0}\SpecialCharTok{:}\DecValTok{12}
\FunctionTok{plot}\NormalTok{(x,}\FunctionTok{dpois}\NormalTok{(x,}\DecValTok{1}\NormalTok{),}\AttributeTok{type=}\StringTok{\textquotesingle{}s\textquotesingle{}}\NormalTok{)}
\FunctionTok{lines}\NormalTok{(x,}\FunctionTok{dpois}\NormalTok{(x,}\DecValTok{2}\NormalTok{),}\AttributeTok{type=}\StringTok{\textquotesingle{}s\textquotesingle{}}\NormalTok{,}\AttributeTok{col=}\DecValTok{2}\NormalTok{)}
\FunctionTok{lines}\NormalTok{(x,}\FunctionTok{dpois}\NormalTok{(x,}\DecValTok{5}\NormalTok{),}\AttributeTok{type=}\StringTok{\textquotesingle{}s\textquotesingle{}}\NormalTok{,}\AttributeTok{col=}\DecValTok{3}\NormalTok{)}
\FunctionTok{legend}\NormalTok{(}\StringTok{"topright"}\NormalTok{,}\AttributeTok{legend=}\FunctionTok{c}\NormalTok{(}\DecValTok{1}\NormalTok{,}\DecValTok{2}\NormalTok{,}\DecValTok{5}\NormalTok{),}\AttributeTok{lty=}\DecValTok{1}\NormalTok{,}\AttributeTok{col=}\DecValTok{1}\SpecialCharTok{:}\DecValTok{3}\NormalTok{)}
\end{Highlighting}
\end{Shaded}

\pandocbounded{\includegraphics[keepaspectratio]{02_Distributions_files/figure-latex/unnamed-chunk-22-1.pdf}}

\begin{Shaded}
\begin{Highlighting}[]
\NormalTok{x }\OtherTok{\textless{}{-}} \DecValTok{20}\SpecialCharTok{:}\DecValTok{80}
\FunctionTok{plot}\NormalTok{(x,}\FunctionTok{dpois}\NormalTok{(x,}\DecValTok{50}\NormalTok{),}\AttributeTok{type=}\StringTok{\textquotesingle{}s\textquotesingle{}}\NormalTok{,}\AttributeTok{ylim=}\FunctionTok{c}\NormalTok{(}\DecValTok{0}\NormalTok{,}\FloatTok{0.08}\NormalTok{))     }\CommentTok{\#Poisson with mean 50 (variance = 50)}
\FunctionTok{lines}\NormalTok{(x}\FloatTok{+0.5}\NormalTok{,}\FunctionTok{dnorm}\NormalTok{(x,}\DecValTok{50}\NormalTok{,}\FunctionTok{sqrt}\NormalTok{(}\DecValTok{50}\NormalTok{)),}\AttributeTok{col=}\DecValTok{2}\NormalTok{)     }\CommentTok{\#Normal with mean and variance of 50}
\FunctionTok{lines}\NormalTok{(x,}\FunctionTok{dbinom}\NormalTok{(x,}\DecValTok{100}\NormalTok{,}\FloatTok{0.5}\NormalTok{),}\AttributeTok{col=}\DecValTok{3}\NormalTok{,}\AttributeTok{type=}\StringTok{\textquotesingle{}s\textquotesingle{}}\NormalTok{)       }\CommentTok{\#Binomial with mean 50 (variance = 25)}
\FunctionTok{legend}\NormalTok{(}\StringTok{"topright"}\NormalTok{,}\AttributeTok{legend=}\FunctionTok{c}\NormalTok{(}\StringTok{"pois"}\NormalTok{,}\StringTok{"norm"}\NormalTok{,}\StringTok{"binom"}\NormalTok{),}\AttributeTok{col=}\DecValTok{1}\SpecialCharTok{:}\DecValTok{3}\NormalTok{,}\AttributeTok{lty=}\DecValTok{1}\NormalTok{)}
\end{Highlighting}
\end{Shaded}

\pandocbounded{\includegraphics[keepaspectratio]{02_Distributions_files/figure-latex/unnamed-chunk-22-2.pdf}}

\begin{Shaded}
\begin{Highlighting}[]
\FunctionTok{plot}\NormalTok{(x,}\FunctionTok{dbinom}\NormalTok{(x,}\DecValTok{100}\NormalTok{,}\FloatTok{0.5}\NormalTok{),}\AttributeTok{type=}\StringTok{\textquotesingle{}s\textquotesingle{}}\NormalTok{,}\AttributeTok{col=}\DecValTok{3}\NormalTok{)        }\CommentTok{\#Binomial with mean 50 (variance = 25)}
\FunctionTok{lines}\NormalTok{(x}\FloatTok{+0.5}\NormalTok{,}\FunctionTok{dnorm}\NormalTok{(x,}\DecValTok{50}\NormalTok{,}\FunctionTok{sqrt}\NormalTok{(}\DecValTok{25}\NormalTok{)),}\AttributeTok{col=}\DecValTok{2}\NormalTok{)     }\CommentTok{\#Normal with mean 50 and variance of 25}
\FunctionTok{lines}\NormalTok{(x,}\FunctionTok{dpois}\NormalTok{(x,}\DecValTok{50}\NormalTok{),}\AttributeTok{col=}\DecValTok{1}\NormalTok{,}\AttributeTok{type=}\StringTok{\textquotesingle{}s\textquotesingle{}}\NormalTok{)         }\CommentTok{\#Poisson with mean 50 (variance = 50)}
\FunctionTok{legend}\NormalTok{(}\StringTok{"topright"}\NormalTok{,}\AttributeTok{legend=}\FunctionTok{c}\NormalTok{(}\StringTok{"pois"}\NormalTok{,}\StringTok{"norm"}\NormalTok{,}\StringTok{"binom"}\NormalTok{),}\AttributeTok{col=}\DecValTok{1}\SpecialCharTok{:}\DecValTok{3}\NormalTok{,}\AttributeTok{lty=}\DecValTok{1}\NormalTok{)}
\end{Highlighting}
\end{Shaded}

\pandocbounded{\includegraphics[keepaspectratio]{02_Distributions_files/figure-latex/unnamed-chunk-22-3.pdf}}

The last two panels depict a comparison of the Poisson, Normal, and
Binomial with the same mean and a large sample size. The Poisson and
Binomial are identical in the two figures but in the first the normal
has the same variance as the Poisson and the second it is the same as
the binomial.

\subsubsection{Questions:}\label{questions-6}

\begin{verbatim}
10)  Normal distributions are often applied to count data even though count data can only take on positive integer values.  Is this fair to do in these two examples? (i.e. how good is the normal approximation)

I think it is fair here. With a high sample size, the normal distibution closely matches the poisson and binomial curves witht he same mean and variance. 

11) Would the normal be a fair approximation to the Poisson curves for small numbers (the first panel)? How about for the Bionomial for small numbers (earlier panel of figures on the Binomial)?

No, I think the normal distibution cuts too far through the steps at low sample sizes. 

12) Is the Poisson a good approximation of the Binomial?

I don't think so, even though the means are the same, the shapes of the curves are quite different. 
\end{verbatim}

\subsection{Negative binomial}\label{negative-binomial}

The negative binomial has two interesting interpretations that are
subtlety different from either the Poisson or Binomial. In the first
case, it is the number of trials needed in order to observe a fixed
number of occurrences, which is the opposite from the Binomial's number
of occurrences in a fixed trial size and thus where it gets its name.
The Negative Binomial also arises as the distribution of number of
events that occur in a fixed space or time when the rate is not constant
(as in the Poisson) but varies according to a Gamma distribution. Hence
the Negative Binomial is also used to describe data that logically seems
to come from a Poisson process but has greater variability that is
expected from the Poisson (which by definition has a variance equal to
its mean). The Geometric distribution arises as a special case of the
negative binomial where the number of occurrences is fixed at 1.

\begin{Shaded}
\begin{Highlighting}[]
\NormalTok{x }\OtherTok{\textless{}{-}} \DecValTok{0}\SpecialCharTok{:}\DecValTok{20}
\DocumentationTok{\#\# negative binomial}

\DocumentationTok{\#\# vary size}
\FunctionTok{plot}\NormalTok{(x,}\FunctionTok{dnbinom}\NormalTok{(x,}\DecValTok{1}\NormalTok{,}\FloatTok{0.5}\NormalTok{),}\AttributeTok{type=}\StringTok{"s"}\NormalTok{,}\AttributeTok{main=}\StringTok{"vary size"}\NormalTok{)}
\FunctionTok{lines}\NormalTok{(x,}\FunctionTok{dnbinom}\NormalTok{(x,}\DecValTok{2}\NormalTok{,}\FloatTok{0.5}\NormalTok{),}\AttributeTok{type=}\StringTok{"s"}\NormalTok{,}\AttributeTok{col=}\DecValTok{2}\NormalTok{)}
\FunctionTok{lines}\NormalTok{(x,}\FunctionTok{dnbinom}\NormalTok{(x,}\DecValTok{3}\NormalTok{,}\FloatTok{0.5}\NormalTok{),}\AttributeTok{type=}\StringTok{"s"}\NormalTok{,}\AttributeTok{col=}\DecValTok{3}\NormalTok{)}
\FunctionTok{lines}\NormalTok{(x,}\FunctionTok{dnbinom}\NormalTok{(x,}\DecValTok{5}\NormalTok{,}\FloatTok{0.5}\NormalTok{),}\AttributeTok{type=}\StringTok{"s"}\NormalTok{,}\AttributeTok{col=}\DecValTok{4}\NormalTok{)}
\FunctionTok{lines}\NormalTok{(x,}\FunctionTok{dnbinom}\NormalTok{(x,}\DecValTok{10}\NormalTok{,}\FloatTok{0.5}\NormalTok{),}\AttributeTok{type=}\StringTok{"s"}\NormalTok{,}\AttributeTok{col=}\DecValTok{5}\NormalTok{)}
\FunctionTok{legend}\NormalTok{(}\StringTok{"topright"}\NormalTok{,}\AttributeTok{legend=}\FunctionTok{c}\NormalTok{(}\DecValTok{1}\NormalTok{,}\DecValTok{2}\NormalTok{,}\DecValTok{3}\NormalTok{,}\DecValTok{5}\NormalTok{,}\DecValTok{10}\NormalTok{),}\AttributeTok{col=}\DecValTok{1}\SpecialCharTok{:}\DecValTok{5}\NormalTok{,}\AttributeTok{lty=}\DecValTok{1}\NormalTok{)}
\end{Highlighting}
\end{Shaded}

\pandocbounded{\includegraphics[keepaspectratio]{02_Distributions_files/figure-latex/unnamed-chunk-23-1.pdf}}

\begin{Shaded}
\begin{Highlighting}[]
\DocumentationTok{\#\# vary probability}
\FunctionTok{plot}\NormalTok{(x,}\FunctionTok{dnbinom}\NormalTok{(x,}\DecValTok{3}\NormalTok{,}\FloatTok{0.5}\NormalTok{),}\AttributeTok{type=}\StringTok{"s"}\NormalTok{,}\AttributeTok{main=}\StringTok{"vary probability"}\NormalTok{)}
\FunctionTok{lines}\NormalTok{(x,}\FunctionTok{dnbinom}\NormalTok{(x,}\DecValTok{3}\NormalTok{,}\FloatTok{0.3}\NormalTok{),}\AttributeTok{type=}\StringTok{"s"}\NormalTok{,}\AttributeTok{col=}\DecValTok{2}\NormalTok{)}
\FunctionTok{lines}\NormalTok{(x,}\FunctionTok{dnbinom}\NormalTok{(x,}\DecValTok{3}\NormalTok{,}\FloatTok{0.2}\NormalTok{),}\AttributeTok{type=}\StringTok{"s"}\NormalTok{,}\AttributeTok{col=}\DecValTok{3}\NormalTok{)}
\FunctionTok{lines}\NormalTok{(x,}\FunctionTok{dnbinom}\NormalTok{(x,}\DecValTok{3}\NormalTok{,}\FloatTok{0.1}\NormalTok{),}\AttributeTok{type=}\StringTok{"s"}\NormalTok{,}\AttributeTok{col=}\DecValTok{4}\NormalTok{)}
\FunctionTok{legend}\NormalTok{(}\StringTok{"topright"}\NormalTok{,}\AttributeTok{legend=}\FunctionTok{c}\NormalTok{(}\FloatTok{0.5}\NormalTok{,}\FloatTok{0.3}\NormalTok{,}\FloatTok{0.2}\NormalTok{,}\FloatTok{0.1}\NormalTok{),}\AttributeTok{col=}\DecValTok{1}\SpecialCharTok{:}\DecValTok{5}\NormalTok{,}\AttributeTok{lty=}\DecValTok{1}\NormalTok{)}
\end{Highlighting}
\end{Shaded}

\pandocbounded{\includegraphics[keepaspectratio]{02_Distributions_files/figure-latex/unnamed-chunk-23-2.pdf}}

\begin{Shaded}
\begin{Highlighting}[]
\DocumentationTok{\#\# vary variance , alternate parameterization}
\NormalTok{mean }\OtherTok{=} \DecValTok{8}
\NormalTok{var }\OtherTok{=} \FunctionTok{c}\NormalTok{(}\DecValTok{10}\NormalTok{,}\DecValTok{20}\NormalTok{,}\DecValTok{30}\NormalTok{)}
\NormalTok{size }\OtherTok{=}\NormalTok{ mean}\SpecialCharTok{\^{}}\DecValTok{2}\SpecialCharTok{/}\NormalTok{(var}\SpecialCharTok{{-}}\NormalTok{mean)}
\FunctionTok{plot}\NormalTok{(x,}\FunctionTok{dnbinom}\NormalTok{(x,}\AttributeTok{mu=}\NormalTok{mean,}\AttributeTok{size=}\NormalTok{size[}\DecValTok{1}\NormalTok{]),}\AttributeTok{type=}\StringTok{"s"}\NormalTok{,}\AttributeTok{ylim=}\FunctionTok{c}\NormalTok{(}\DecValTok{0}\NormalTok{,}\FloatTok{0.14}\NormalTok{),}\AttributeTok{main=}\StringTok{"vary variance"}\NormalTok{)}
\FunctionTok{lines}\NormalTok{(x,}\FunctionTok{dnbinom}\NormalTok{(x,}\AttributeTok{mu=}\NormalTok{mean,}\AttributeTok{size=}\NormalTok{size[}\DecValTok{2}\NormalTok{]),}\AttributeTok{type=}\StringTok{"s"}\NormalTok{,}\AttributeTok{col=}\DecValTok{2}\NormalTok{)}
\FunctionTok{lines}\NormalTok{(x,}\FunctionTok{dnbinom}\NormalTok{(x,}\AttributeTok{mu=}\NormalTok{mean,}\AttributeTok{size=}\NormalTok{size[}\DecValTok{3}\NormalTok{]),}\AttributeTok{type=}\StringTok{"s"}\NormalTok{,}\AttributeTok{col=}\DecValTok{3}\NormalTok{)}
\FunctionTok{legend}\NormalTok{(}\StringTok{\textquotesingle{}topright\textquotesingle{}}\NormalTok{,}\AttributeTok{legend=}\FunctionTok{format}\NormalTok{(}\FunctionTok{c}\NormalTok{(var,mean),}\AttributeTok{digits=}\DecValTok{2}\NormalTok{),}\AttributeTok{col=}\DecValTok{1}\SpecialCharTok{:}\DecValTok{4}\NormalTok{,}\AttributeTok{lty=}\DecValTok{1}\NormalTok{)}
\FunctionTok{lines}\NormalTok{(x,}\FunctionTok{dpois}\NormalTok{(x,mean),}\AttributeTok{col=}\DecValTok{4}\NormalTok{,}\AttributeTok{type=}\StringTok{"s"}\NormalTok{)}
\end{Highlighting}
\end{Shaded}

\pandocbounded{\includegraphics[keepaspectratio]{02_Distributions_files/figure-latex/unnamed-chunk-23-3.pdf}}

\begin{Shaded}
\begin{Highlighting}[]
\DocumentationTok{\#\# NB as generalization of pois with inflated variance}

\DocumentationTok{\#\# geometric}
\FunctionTok{plot}\NormalTok{(x,}\FunctionTok{dgeom}\NormalTok{(x,}\FloatTok{0.5}\NormalTok{),}\AttributeTok{type=}\StringTok{"s"}\NormalTok{,}\AttributeTok{main=}\StringTok{"Geometric"}\NormalTok{)}
\FunctionTok{lines}\NormalTok{(x,}\FunctionTok{dgeom}\NormalTok{(x,}\FloatTok{0.15}\NormalTok{),}\AttributeTok{type=}\StringTok{"s"}\NormalTok{,}\AttributeTok{col=}\DecValTok{2}\NormalTok{)}
\FunctionTok{lines}\NormalTok{(x,}\FunctionTok{dgeom}\NormalTok{(x,}\FloatTok{0.05}\NormalTok{),}\AttributeTok{type=}\StringTok{"s"}\NormalTok{,}\AttributeTok{col=}\DecValTok{3}\NormalTok{)}
\FunctionTok{lines}\NormalTok{(x,}\FunctionTok{dnbinom}\NormalTok{(x,}\DecValTok{1}\NormalTok{,}\FloatTok{0.15}\NormalTok{),}\AttributeTok{type=}\StringTok{"s"}\NormalTok{,}\AttributeTok{col=}\DecValTok{4}\NormalTok{,}\AttributeTok{lty=}\DecValTok{2}\NormalTok{)}
\end{Highlighting}
\end{Shaded}

\pandocbounded{\includegraphics[keepaspectratio]{02_Distributions_files/figure-latex/unnamed-chunk-23-4.pdf}}

\begin{Shaded}
\begin{Highlighting}[]
\DocumentationTok{\#\# geometric as special case of NB where size = 1}
\end{Highlighting}
\end{Shaded}

\subsubsection{Questions:}\label{questions-7}

\begin{verbatim}
13)  In the 'vary size' panel, what are the means of the curves? 

The means are the same as the size in that panel. 

14) In the “vary variance” panel, how does the shape of the Negative Binomial compare to a Poisson with the same mean?

At a small variance, the negative binomial looks very similar to poisson. At high variances, the negative binomial moves closer to 0 and has thicker tails than poisson.
\end{verbatim}

\emph{AI Statement: AI (CoPilot) was used to understand questions and
keep track of what single letter abbreviations (n,p,r,a, b\ldots) and
their meaning. Never to generate code or answers.}

\end{document}
